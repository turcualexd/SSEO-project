\section{Main mission phases}
\label{sec:phases}

The Juno mission was divided into five phases: LEOP, Cruise, Jupiter approach and insertion, Science Operations and De-orbit.
\begin{enumerate}
    \item \textbf{LEOP}
    
    % the juno launch
    Following the launch from Cape Canaveral, the spacecraft entered a low Earth parking orbit. \cite{Juno_launch}
    Afterwards Juno was injected in an interplanetary trajectory and was separated from its upper stage after SECO-2 at time L+54 min.
    The solar panels deployment was performed about five minutes after the spacecraft separation, and it took approximately five minutes.

    \item \textbf{Cruise}
    
    %At the beginning of this phase, the spacecraft was injected in an interplanetary trajectory with the aim of reaching Jupiter. 
    The cruise had a duration of about five years, during which two deep space manoeuvres, multiple minor corrections and an Earth fly-by were performed.
    All manoeuvres will be better described in \autoref{sec:mission_analysis}. This phase also included instruments testing and verification, to ensure they were functioning properly and ready for the usage during the mission. 

    \item \textbf{Jupiter approach and insertion}
    
    This phase began four days before the start of orbit insertion manoeuvre and ended one hour after the start of the orbit insertion manoeuvre. The latter occurred at closest approach to Jupiter and slowed the spacecraft down enough to let it be captured by Jupiter in a 53-days period orbit.
    The Jupiter orbit insertion burn was performed by the Leros 1-b main engine, and it lasted 30 minutes. After the burn, the spacecraft was in a polar orbit around Jupiter.
    The 53-days orbit provided substantial propellant savings with respect to the direct insertion in the operational orbit.

    \item \textbf{Science operations}
    
    The Juno polar and highly eccentric orbit was designed to facilitate the close-in measurements and to minimize the time spent in the Jupiter radiation belts. During this phase all the science operations, in different attitudes, are being performed.

    \item \textbf{De-orbit}
    
    The de-orbit phase will occur during the final orbit of the mission. The latter was designed to satisfy NASA's planetary protection requirements and ensure that Juno doesn't impact any of Jupiter's moons. A de-orbit burn will be performed, placing the spacecraft on a trajectory towards Jupiter inner and denser layer of the atmosphere where it will burn up.   
     
\end{enumerate}