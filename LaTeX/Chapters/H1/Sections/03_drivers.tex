\section{Mission drivers}
\label{sec:drivers}

%\textcolor{red}{
    The drivers of a mission are identified as critical requirements that lead completely or partly the design process of one or more subsystems.%}
     Being Juno an interplanetary mission starting from a distance of around 1 AU, with a final nominal distance from the Sun of 5.2 AU, and operating in an highly radiation intense environment, the following drivers have been identified: 

\begin{enumerate}
    %\color{red}
    \item \textbf{Insertion on Jupiter’s orbit} \cite{fact_sheet} \cite{video_1h} \cite{Juno_launch} \cite{Key_requirements} 
    
    One of the most critical phases to achieve the mission purpose is the insertion in Jupiter’s orbit. The criticality is related to the high complexity of lowering the energy to get captured by the planet. To slow down and enter Jupiter’s orbit, the spacecraft has to perform a delicate maneuver, otherwise it flies off into space thus science mission fails. As a matter of fact, Juno’s engines shall fire at the right moment in the correct direction for the required amount of time. 

    \item \textbf{Surviving in the Jupiter environment} \cite{fact_sheet} \cite{2006_overview} \cite{atmosphere}
    
    The Juno mission shall orbit around Jupiter. The sunlight received by the planet is 25 times less than the one received by Earth, furthermore Jupiter is characterized by an enormous magnetic field which is nearly 20,000 times as powerful as Earth’s field and huge intensity of radiations. As a matter of fact, Juno shall benefit the usage of thermal blankets as well as radiation-shielded electronics vault, avoid to fly over the most of Jupiter’s high-radiation regions and increase the tech-quality of cells by composing solar arrays and increasing panels’ area.
   
    \item \textbf{Maintaining communication during the journey and the science operations}\cite{Juno_launch}\cite{communication_support}
    
    Due to the huge distance and high interferences caused by the jovian radiation field and interplanetary radiations, communication and data handling shall be considered as a driver. Referring to the Juno’s high level goals, the mission success requires collecting science data and safely transmitting back to Earth. To ensure the latter, Juno shall include a flight processor designed to operate in a strong radiation environment and support an amount of total instruments throughput sufficient for the payload suite. Relatively to the second purpose, Juno shall establish communication with ground station alongside gathering scientific intel via a HGA that supports X-band communications with Earth for command uplink and science data and telemetry downlink. Communications with Juno shall be accomplished by NASA’s Deep Space Network Station.
    Juno shall also include low and medium gain communication system to provide continuous communication, even when the HGA is not pointing towards Earth.

\end{enumerate}