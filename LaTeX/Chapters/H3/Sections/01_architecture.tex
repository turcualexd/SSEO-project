\section{TMTC architecture}
\label{sec:TMTC_architecture}

\rfig{antennas_positioning}{0.55}{Location of telecommunication antennas}

% cosa fa il systema di tmtc, a che banda opera. quante antenne ha.
% perchè serve una doppia banda per comunicare ( X + Ka )

The Juno TMTC subsystem purpose is to communicate data about the status of the spacecraft, download scientific data and to receive commands to and from the DSN antennas. Both the uplink and downlink are performed in X-band frequency: 7.15 GHz the downlink and 8.40 GHz the uplink. One of the main goals of the mission is to study Jupiter's gravity field: this is accomplished by exploiting the difference in doppler effect of the telecommunication from the model and the real jovian gravity field. Due to the harsh environment that Juno faces and the need to measure precisely the residual frequency, transmission on both X-band Ka-band during gravity science is needed. 
For this reason, the HGA can operate in 3 different modes: 

\begin{itemize}
\item \textbf{X/X}: uplink and downlink are coherent and performed in X-band;
\end{itemize}

\begin{itemize}
\item \textbf{X/X \& X/Ka}: simultaneous transmission on X-band (uplink and downlink), together with a coherent Ka-band downlink at 32 GHz and X-band uplink;
\item \textbf{X/X \& Ka/Ka}: phase coherent X-band uplink and downlink together with a phase coherent Ka-band uplink at 34 GHz and downlink.
\end{itemize}
Five antennas are mounted onboard Juno with different orientations, positions and capabilities: one HGA, one MGA, two LGAs and one TLGA.

% parlare dei sui transponder

\subsection{HGA}
\label{subsec:hga}

% doppio scopo: comunicare e scienza. frequenze, dimensioni, come è fatta

The HGA is the principal means of communications with Earth throughout most of the cruise and science mission. It is mounted on the forward deck, aligned with the spin axis of the spacecraft, as shown in \autoref{fig:antennas_positioning}. Due to the significant distance between Juno and DSN antennas and the limited transmitter power, HGA gain maximization was a priority. Constrains on dish dimension and in the attitude control of the spin stabilized spacecraft were present: the Atlas V fairing limited the HGA dish diameter to 2.5 meters, then the presence of massive solar arrays prohibited the ability to point the main beam to anything tighter than about $\pm$ 0.25°. The latter limitation would have led to an insufficient gain in the HGA, preventing closing the link with Earth. The most limiting factor in designing the HGA was the need of both transmitting and receiving on X-band and Ka-band without limiting the performance of any signal. Because of these requirements, a dual reflector Gregorian style optics was installed. The latter consists in a parabolic main reflector and an elliptical sub-reflector, making the whole system low mass and compact. To obtain 
 
\subsection{MGA}
\label{subsec:mga}

\subsection{LGA}
\label{subsec:lga}

\subsection{TLGA}
\label{subsec:tlga}

\subsection{Ground stations}


