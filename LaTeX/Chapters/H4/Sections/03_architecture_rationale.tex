\section{Architecture and rationale of AOCS}
\label{sec:AOCS_architecture_rationale}



\subsection{Sensors}
\label{subsec:Sensors}

Juno's AOCS employs the following main attitude sensors:

\begin{itemize}
    \item \textbf{2 Stellar Reference Units (SRUs)} custom built by Selex Galileo (now Leonardo S.p.A.) mounted on the forward deck of the spacecraft facing radially outwards. These units are based on the A-STR \cite{SRU}, modified with further radiation shielding to survive the harsh environment of Jupiter, bringing the total weight up to 7.8 kg. One of the most important characteristics of these sensors is the ability to operate in a Time Delay Integration (TDI) mode, which allows them to compensate for the spin of the spacecraft when capturing an image. Other specifications of the standard A-STR are reported in \autoref{table:star_sensors}.
    \begin{table}[H]
        \renewcommand{\arraystretch}{1.3}
        \centering
        \begin{tabular}{|c|c|c|c|}
            \hline
            \textbf{FOV [deg]} & \textbf{Update rate [Hz]} & \textbf{Bias Error [arcsec]} & \textbf{FOV error [arcsec]} \\
            \hline
            16.4 $\times$ 16.4 & 4 or 10 & \makecell{8.25 (pitch/yaw) \\ 11.1 (roll)} & \makecell{$<$ 3.6 (pitch/yaw) \\ $<$ 21 (roll)} \\   
            \hline
        \end{tabular}

        \vspace{5mm}

        \begin{tabular}{|c|c|c|c|}
            \hline
            \textbf{Size (L $\boldsymbol{\times}$ W $\boldsymbol{\times}$ H) [cm]} & \textbf{Mass [kg]} & \textbf{Power consumption [W]} & \textbf{Operational temperature [°C]} \\
            \hline
            19.5 $\times$ 17.5 $\times$ 29.1 & 3.55 & \makecell{8.9 @ 20°C \\ 13.5 @ 60°C} & -30 to +60\\
            \hline
        \end{tabular}
        \caption{A-STR specifications}
        \label{table:star_sensors}
    \end{table}
 
    \item \textbf{2 Spinning Sun Sensors (SSSes)} by Adcole Maryland Aerospace \cite{SSS}, also positioned on the edge of the forward deck with a similar orientation as the SRUs. They are specialized for attitude determination on a spinning spacecraft and allow for a fail safe recovery. Sun sensors are required during ME burns, when SRUs might be pointing the Sun. Useful specifications are shown in \autoref{table:sun_sensors}.
    
    % Assumiamo siano richiesti per proteggere le sru durante le main engine burn quando Juno non punta la terra o il sole e quindi le sru potrebbero puntare al sole
    \begin{table}[H]
        \renewcommand{\arraystretch}{1.3}
        \centering
        \begin{tabular}{|c|c|c|}
            \hline
            \textbf{FOV [deg]} & \textbf{Size of sensor (L $\boldsymbol{\times}$ W $\boldsymbol{\times}$ H) [cm]} & \textbf{Size of electronics (L $\boldsymbol{\times}$ W $\boldsymbol{\times}$ H) [cm]} \\
            \hline
            $\pm$ 64 & 6.6 $\times$ 3.3 $\times$ 2.5 & 5.1 $\times$ 8.2 $\times$ 8.9 \\   
            \hline
        \end{tabular}

        \vspace{5mm}

        \begin{tabular}{|c|c|c|c|}
            \hline
            \textbf{Accuracy [deg]} & \textbf{Mass of sensor [kg]} & \textbf{Mass of electronics [kg]} & \textbf{Power consumption [W]}\\
            \hline
            \makecell{$\pm$ 0.1 at 0° \\  $\pm$ 0.6 at 64°} & 0.109 & 0.475 to 0.725 & 0.4\\
            \hline
        \end{tabular}
        \caption{SSSes specifications}
        \label{table:sun_sensors}
    \end{table}
 
    
    \item \textbf{2 Inertial Measurement Units (IMUs)} by Northrop (Hypothesizing heritage from Cassini \cite{gyro_evaluation} and MESSENGER \cite{messenger_imu}) inside Juno's main body. One of their biggest advantage is utilizing Hemispherical Resonator Gyroscopes (HGRs) which, due to their construction and inner workings, are inherently radiation hardened and highly resistant to wear and ageing. In particular Scalable Space Inertial Reference Units (SSIRUs) \cite{SSIRU} are used which were also modified for this specific mission like the A-STR. Their nominal specifications are shown in \autoref{table:SSIRU}.
     
    \begin{table}[H]
        \renewcommand{\arraystretch}{1.3}
        \centering
        \begin{tabular}{|c|c|c|}
            \hline
            \textbf{Power [W]} & \textbf{Size of sensor (L $\boldsymbol{\times}$ W $\boldsymbol{\times}$ H) [cm]} & \textbf{Weight [kg]} \\
            \hline
            43 max & 28.9 $\times$ 18 $\times$ 14.9 & 7.1 \\   
            \hline
        \end{tabular}

        \vspace{5mm}

        \begin{tabular}{|c|c|c|}
            \hline
            \textbf{Operational temperature [°C]} & \textbf{Angle Random Walk [$\textrm{deg} / \sqrt{\textrm{hr}}$]} & \textbf{HAMRR [deg/s]} \\
            \hline
            \makecell{-55 to +85 (non-operational) \\  -10 to +60 (full performance)} & < 0.00015 & $\pm$ 7 \\
            \hline
        \end{tabular}
        \caption{SSIRUs specifications}
        \label{table:SSIRU}
    \end{table}
\end{itemize}

All of this sensors are doubled to provide cold redundancy, meaning that only one unit is powered during nominal operations while the other one is switched off.

An additional sensor, the \textbf{Advanced Stellar Compass (ASC)}, is present on Juno to support the MAG experiment suite. It's comprised of four \textbf{Camera Head Units (CHUs)}, two per each FGM for redundancy, mounted on the MAG boom pointing towards the -Z direction and inclined by $\pm$ 13° along the Y-axis. Their objective is to achieve an even more precise attitude determination near the instrument location with the help of the SRUs, which also function as a backup in case the ASC fails. These sensors were designed and built by the Technical University of Denmark (DTU) as largely off-the-shelf products. \cite{ASC_details}

%are present on Juno, to allow a precise attitude determination for magnetic sensors.

\subsection{Actuators}
\label{subsec:Actuators}

As previously described in the analysis of the propulsion system (Homework 2) Juno utilizes twelve MR-111C RCS thrusters by Aerojet Rocketdyne \cite{RCS_info} divided into four groups of three. Each set is housed on a Rocket Engine Module (REM) on top of four pylons, two on the forward deck and two on the aft deck, extending in the Z-axis and mounted along the Y-axis. 
\cfig{foto_thruster}{0.7}{RCS mount direction}  
The pylons are raised respectively by 74 cm on the forward deck and about 26 cm on the aft deck. As shown in \autoref{fig:foto_thruster} each cluster includes an axial thruster, denoted by the letter "A", and two lateral ones, denoted as "L".
Axial thrusters are canted 10° away from the Z-axis while the lateral thrusters are canted 5° away from the X-axis and 12.5° toward the Z-axis \cite{juno_inner}. The specifics of the MR-111C thrusters are presented in \autoref{table:RCS_specifications}. \cite{RCS_values}

\begin{table}[H]
    \renewcommand{\arraystretch}{1.3}
    \centering
    \begin{tabular}{|c|c|c|}
        \hline
        \textbf{Thrust [N]} & \textbf{Specific impulse [s]} & \textbf{MIB [Ns]} \\
        \hline
        4.5 & 220 & 0.08 \\   
        \hline
    \end{tabular}

    \vspace{5mm}

    \begin{tabular}{|c|c|c|c|}
        \hline
        \textbf{Propellant} & \textbf{Catalyst} & \textbf{Mass [kg]} & \textbf{Power usage [W]}\\
        \hline
        Hydrazine & S-405 & 0.33 & 13.64 \\ 
        \hline
    \end{tabular}
    \caption{MR-111C specifications}
    \label{table:RCS_specifications}
\end{table}

Additional hardware is present on the spacecraft to aid attitude control, while not being full-fledged actuators. In particular the supporting struts of the three solar arrays can be moved to adjust their position in order to align the principal inertia axis with the geometrical Z-axis. An active nutation damper is also installed on-board, able to reduce unwanted nutation by generating a controlled damping torque. Given the stringent pointing requirements an active system was chosen for its higher damping rate.   

\subsection{Rationale}
\label{subsec:Rationale}

All AOCS components were mainly chosen due to their high TRL, like in all other systems of the spacecraft. This due both to the complexity of the mission and the harsh environment of Jupiter. Further criteria for the selection and positioning of each unit can be highlighted:
\begin{itemize}
    \item \textbf{SRUs:} star sensors are the most accurate and are capable of complete AD independently of any other sensor. Moreover magnetometers weren't really taken into consideration due to the lack of an accurate model of Jupiter's magnetic field. This leaves sun sensors and IMUs as possible options but neither could be used as the main source of AD due respectively to low accuracy or the need to be periodically realigned. SRUs positioning is dictated by the need of leaving their FOV unobstructed.
    \item \textbf{SSSes:} sun sensors are still needed since during ME burns since the orientation of Juno during this manoeuvre could point the SRUs towards the Sun, possibly damaging them. The SSSes main goal is thus to detect the Sun before it gets too close to the SRUs FOV, in order to shut them when needed. This also explains their proximity to the SRUs.
    \item \textbf{IMUs:} since the sun sensors aren't capable of a complete AD while the SRUs aren't being used, another component able to do so is required and IMUs were the only remaining choice. They are also employed during large precessions (larger than $\sim$2.5°) and required for active nutation damping and spin control.
    \item \textbf{RCS Thrusters:} thrusters were chosen since they are able to provide a high control torque while remaining compact and integrated with the propulsion system. On the contrary reaction wheels and CMG would take up much more space to generate the same control action, thus being less space efficient and cumbersome. Magnetorquers, instead, weren't even considered for the same reasons as magnetometers. Regarding the thrusters' positioning, their orientation presented in \autoref{subsec:Actuators} are due to the need of limiting the interaction of the exhaust gasses with the on board instruments, the HGA and solar arrays.     
\end{itemize}  
