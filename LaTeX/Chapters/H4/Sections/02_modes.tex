\section{Breakdown of Juno modes}
\label{sec:AOCS_modes}

Throughout the mission phases, Juno has to accomplish different tasks through various modes. The principal ones that have been identified are here described.

\subsection{Sun Pointing Mode (SPM)}
\label{subsec:sun_pointing_mode}

This mode aims at pointing the spin axis of S/C (+Z axis) to the Sun. This mode is used in order to:
\begin{itemize}
    \item keep the solar panels pointed to the Sun to provide energy; this is even more crucial in the initial phases of the mission, when all the system checkouts have to be performed;
    \item thermally protect the satellite's vault using the HGA as a shield when the satellite is relatively close to the Sun.
\end{itemize}

This mode is applied mainly when the SPE angle is too large to ensure the communication through the HGA. This condition occurs during the first phases of the mission and nearby the EGA, when the satellite is in proximity of Earth and relatively close to the Sun. The LGAs are therefore used during this mode to communicate with ground.

The pointing requirement for this mode is not very strict since the solar panels are sized for much higher distances and can provide enough energy, the LGAs have a wide beamwidth to communicate with ground and the HGA is large enough to protect the vault.
The spin rate of the satellite is set to $1$ RPM for this interplanetary mode. This improves the passive stability of the pointing, reducing the burden on the active attitude control.

\subsection{Earth Pointing Mode (EPM)}
\label{subsec:earth_pointing_mode}

The EPM is a contiguous interplanetary mode to the SPM, the +Z axis of the S/C is pointed to Earth.
It triggers when the distance from the Sun is high enough to ensure a safe thermal dissipation without the aid of the HGA ($\approx 1.4$ AU) \mref.
Moreover, the S/C will be far enough from Earth to ensure both HGA communications and the fulfillment of power requirements, hence the SPE is low.

\subsection{GRAVity science Mode (GRAVM)}
\label{subsec:grav_mode}

\subsection{MicroWave Radiometer Mode (MWRM)}
\label{subsec:mwr_mode}

\subsection{Turn-Burn-Turn Mode (TBTM)}
\label{subsec:tbt_mode}

\subsection{VECtor Mode (VECM)}
\label{subsec:vec_mode}

\subsection{Spin Change Mode (SCM)}
\label{subsec:spin_change_mode}

\subsection{Safe Modes}
\label{subsec:safe_modes}