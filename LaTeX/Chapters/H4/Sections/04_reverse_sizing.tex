\section{Reverse sizing of AOCS}
\label{sec:AOCS_sizing}

In this section a reverse sizing of the AOCS is performed. The complexity of the dynamics of a spin-stabilized satellite only controlled via RCS required the use of a Simulink model, specifically developed for this assignment and presented in \hyperref[subsec:simulink]{appendix~\ref*{subsec:simulink}}.

\subsection{Modeling hypothesis}
\label{subsec:mod_hypo}
The preliminary reverse sizing process is based on some simplifying assumptions:

\begin{itemize}
    \item
    \begin{minipage}[t]{0.723\linewidth}
        \textbf{Geometry}: Juno spacecraft has been modeled via SolidWorks software, simplifying its shapes but conserving the general dimensions, in particular the size of the solar panels, the vault and the central body. The mass of the system is assumed to be constant throughout the whole mission, at 3625 kg, so the propellant mass needed is overestimated.
        % dimensioni dei pannelli, dimensioni del corpo centrale, massa, inerzie 
        Principal moments of inertia, calculated from the centre of mass, are assumed to be aligned with the geometric axes of the model: this assumption is pretty compliant with the real satellite dynamics thanks to the actuators mounted on solar panel as described in \autoref{subsec:Actuators}. 
       
        The central body was designed as an hexagonal prism with an height of 1.48 m and sides of 1.79 m, the three solar arrays as thin panels of 8 m $\times$ 2.7 m $\times$ 0.03 m, one of them shorter at 6 m due to the presence of the MAG boom, the vault as a rectangular prism of 0.8 m $\times$ 0.8 m $\times$ 0.7 m and the HGA as a rotational ellipsoid with a major semi-axis of 1.25 m and a minor semi-axis of 0.625 m. 
        A view of the CAD model is presented in \autoref{fig:juno_cad}.
    \end{minipage}\hfill
    \begin{minipage}[t]{0.277\linewidth}
        \vspace*{-10mm}
        \cfig{juno_cad}{0.8}{Juno CAD model}
    \end{minipage}
    %\pagebreak
    \item \textbf{Sensors}: the attitude of the spacecraft is assumed to be correct at all times. No errors are present as nor SRUs nor IMUs nor SSSes were modeled. 
    
    \item \textbf{Thrusters}: all twelve thrusters are decoupled one from the other. This assumption changes the dynamics of the spacecraft as the real satellite has to perform consecutive burns via RCS in VECM while in the model only one burn is required. 
    Different arms for each thruster were considered as they are not symmetrical with respect to the centre of mass: 2.7 m for thrusters controlling X and Y-axis, 3.2 m for thrusters controlling the Z-axis.

    \item \textbf{Controller}: the performance of the various manoeuvres is influenced by the chosen control law. Parameters such as maximum angular speed, acceleration and total time of the manoeuvres are checked to verify the obtained results. 
    
    % dire perchè non parliamo delle perturbazioni nella quei cruise 
    \item \textbf{Phases}: only two phases have been simulated, each with its own set disturbances: IC from L+3d to L+822d and jovian planetary phase. 
    OC was not considered as SRP lowers and no other significant disturbances act on Juno. However, the same amount of fuel consumed during ICs will be allocated for the OC to point Earth for telecommunications.
    For the IC phase three different sections are considered: IC 1, IC 2 and IC 3. Relations between these sections and different modes are shown in \autoref{fig:phases_modes}.
    Sun pointing has been considered for all three sections as a worst case scenario sizing, even if IC 2 is in Earth pointing mode (Earth and Sun are almost aligned in this section so the reasoning is not that far off). During each section of the IC, the SRP is considered constant and equal to the integral mean value calculated on the length of the section.
    For the correction manoeuvres the following scheme was implemented: the spin rate is continuously controlled while the spin axis is realigned every 20 days through a 20 minutes correction manoeuvre. This saves a considerable amount of propellant with respect to an uninterrupted control of both attitude and angular speed while always remaining compliant with the constraints shown in \autoref{table:pointing_budget}. 
    Slew manoeuvres are also considered to take into account the movement of the spacecraft relative to the Sun as the pointing cannot be considered inertial. The simulation takes into account only one trial and assumes that no differences are present between contiguous sections. Realignments of Juno's axes can be related, in general, to correction manoeuvres as the system is able to handle different disturbances from the nominal orbit. 
    
    Jupiter planetary phase takes into account SRP, magnetic disturbances and gravity gradient (GG). An 11-days elliptical orbit was considered for the simulation and repeated 33 times as required by the nominal mission. 

    \item \textbf{Manoeuvres}: two different kinds of manoeuvres were modeled: slew manoeuvres and SCMs. The first ones are executed at each DSM, where the TBTM is employed. A worst case scenario has been identified, within the DSMs requirements, with a change of 90° in angular momentum orientation. Observed rate of this manoeuvre from Nasa Eyes\cite{nasa_eyes} shows an angular velocity of about 0.1 °/s. This will be used as the maximum value allowed. 
    
    To simulate SCM only the component along the main spin axis (Z-axis) is controlled and no attitude corrections are performed.      

    %\item \textbf{Fuel consumption}: this value has been calculated considering each thruster firing and its respective duration over an entire phase. 

\end{itemize}

\subsection{Perturbations}
\label{subsec:perturbations}

There are four attitude perturbations that need to be analyzed: solar radiation pressure (SRP), gravity gradient (GG), magnetic and aerodynamic disturbances. In the harsh environment of Jupiter some of those have a significant impact on the spacecraft, while others could be neglected. 

\begin{itemize}
    \item \textbf{Magnetic Torque}: this disturbance was considered only around Jupiter. The model used to describe its magnetic field consists in a dipole modeled around the work of Acuña et al\cite{jupiter_mag_field}, and its value is shy of $4.3\cdot 10^{-4}$ T. Juno's magnetic dipole was assumed from literature, considering a high value of $0.05 \; \textrm{Am}^2$ as a worst case scenario.

    \item \textbf{SRP}: this disturbance is considered both during the IC and the jovian phase. However, in the latter phase it has a less relevant effect as the intensity of the radiation coming from the Sun goes as the inverse of the distance squared. Values considered in the different phases and sections are reported in \autoref{table:SRP_GGG}. 
    As can be seen in the first row of this table, solar radiation is higher during IC 1 and 3, where the spacecraft is closer to the Sun. 
    
    SRP applies a force on different surfaces: the cross section of Juno that faces the Sun is 70 $\textrm{m}^2$. As a first approximation this value was divided in three equal parts, as the number of solar arrays: the torque is computed by considering the SRP acting in the barycenter of each panel. The large area considered made the SRP the dominant perturbation throughout the whole IC phase, despite of the distance from the Sun. The total torque is only applied to the Y-axis due to the geometry of the spacecraft. Reflectivity was assumed as 0.55 for all surfaces facing the Sun, while the arms for the torque were calculated from the CAD model.

    \item \textbf{Gravity Gradient}: this disturbance is considered only during the planetary phase around Jupiter. As can be seen from the last row of \autoref{table:SRP_GGG}, it is orders of magnitude lower than the SRP during the other phases. The torque considered for the various ICs is derived from the average distance between Juno and the Sun, while the value for the jovian phase is reported as the maximum found around the whole orbit. To avoid considering only the worst case, the dynamics of this perturbation was also modeled during a nominal 11-days orbit.
    
    \begin{table}[H]
        \renewcommand{\arraystretch}{1.3}
        \centering
        \small
        \begin{tabular}{|c|c|c|c|c|}
            \hline
            &\textbf{IC 1} & \textbf{IC 2} & \textbf{IC 3} & \textbf{Jovian phase} \\
            \hline
            \hline
            $F_{s} \; [W/m^2]$ & 1225 & 413 & 1339 & 4.08 $\cdot 10^{-2}$ \\
            \hline
            $Torque_{SRP}$ [Nm] & 0.991 & 0.33 & 1.08 & 0.04 \\
            \hline
             $Torque_{GG}$ [Nm] & $3.5 \cdot 10^{-10}$ & $5.4 \cdot 10^{-11}$ & $3.5 \cdot 10^{-10}$ & $5.39 \cdot 10^{-4}$ \\
            \hline
        \end{tabular}
        \caption{SRP and Gravity Gradient relevant values}
        \label{table:SRP_GGG}
    \end{table}
    \vspace*{-4mm}
    
    \item \textbf{Atmospheric Drag}: this disturbance is meaningful only in case of a dense atmosphere. During the IC phase Juno is in vacuum, so no atmosphere is present. For the planetary phase, instead, Jupiter's atmosphere needs to be evaluated: at 1000 km of altitude, density is already in the order of $10^{-11}$ $\textrm{kg/m}^3$ and Juno's closest approach is above 4000 km. Supposing an exponential decay of the density with altitude (as in Earth's atmospheric model) the value of the density allows to neglect the atmospheric drag\cite{jupiter_density}.
    
        
\end{itemize}

\subsection{Propellant reverse sizing}
\label{subsec:prop_rev_sizing}

Results of the control action in various scenarios are analyzed in this section.
\begin{itemize}
    \item \textbf{SCM}: this control mode is employed 10 times to change the angular speed of the spacecraft in order to perform both ME burns and science operations. Nominally Juno is spinning at 1 RPM during every cruise phase, 2 RPM during science operations and Fly-by and at 5 RPM during all ME burns. A total of 10 manoeuvres of this kind were performed, 5 spin-ups and 5 spin-downs, distributed along the whole mission.  A total of 33.14 kg of hydrazine was found to be consumed. A maximum acceleration of 0.05°$\textrm{/s}^2$ was observed. In \autoref{table:spin} the amount of fuel utilized for each single change in spin is reported together with the number of times it had to be performed, in one way or the other. Spin changes reported in the first column were performed before and after DSMs and before JOI; the ones in the second column were executed before and after the Fly-by; the ones in the last column were done after the JOI and before and after the PRM.
    
    \vspace*{-2mm}
    \begin{table}[H]
        \renewcommand{\arraystretch}{1.3}
        \centering
        \small
        \begin{tabular}{|c|c|c|c|c|}
            \hline
            &\textbf{1 RPM $\leftrightarrow$ 5 RPM } & \textbf{1 RPM $\leftrightarrow$ 2 RPM } &\textbf{2 RPM $\leftrightarrow$ 5 RPM} & \textbf{Total} \\
            \hline
            \hline
            Fuel Consumption [kg] & 4.27 & 1.07 & 3.21 & 33.14 \\
            \hline
            \# of occurrences & 5 & 2 & 3 & 10 \\
            \hline
        \end{tabular}
        \caption{Spin change manoeuvres}
        \label{table:spin}
    \end{table}
    \vspace*{-4mm}

    \item \textbf{Slew manoeuvre}: this specific manoeuvre is performed eight times to align the ME with the required direction to perform a burn (\autoref{subsec:tbt_mode}): twice for each DSM while spinning at 1 RPM, twice for the JOI (once spinning at 1 RPM and once spinning at 2 RPM) and twice for the PRM while spinning at 2 RPM. A worst case scenario is always considered with a 90° realignment. \cite{LL_early_cruise} 
    All the slew manoeuvres are performed with a maximum velocity of 0.1 °/s as stated in \autoref{subsec:mod_hypo}. 
    Given the considerable moment of inertia along the Z-axis, a higher rotational speed implies a significant augment in fuel consumption. Results are of $2.7$ kg for each slew at 1 RPM and $6.28$ kg for each slew at 2 RPM. Total consumption for these manoeuvres is $32.34$ kg. 
    
    \item \textbf{Interplanetary Phase corrections}: in this control mode only the SRP is considered affecting both angular speed and attitude of the spacecraft. Corrections of the angular velocity are continuously performed to ensure stability in pointing without correcting the attitude directly. However, after a 20-days period, corrections are needed to align the HGA to its nominal pointing requirement. \cite{LL_early_cruise}. All the constraints cited in \autoref{subsec:pointing_budget} are respected. 
    The consumption of hydrazine in these phases is reported in \autoref{table:cruise}.

    \vspace*{-2mm}
    \begin{table}[H]
        \renewcommand{\arraystretch}{1.3}
        \centering
        \small
        \begin{tabular}{|c|c|c|c|c|}
            \hline
            &\textbf{IC 1} &\textbf{IC 2} & \textbf{IC 3} &\textbf{Total}\\
            \hline
            \hline
            Fuel Consumption [kg] & 4.82 & 3.00 & 13.94 & 21.76 \\
            \hline
        \end{tabular}
        \caption{IC consumption}
        \label{table:cruise}
    \end{table}
    \vspace*{-4mm}

    \item \textbf{Jovian planetary Phase}: while orbiting Jupiter, Juno performs science spinning at 2 RPM, ensuring natively higher robustness to disturbances. Major corrections must be performed near the pericentre of the orbit, as its speed approaches 58 km/s.  
    Requirements in pointing have been satisfied in this phase as the maximum error achieved during the simulation was lower than 0.025°.  Consumption of hydrazine is estimated at $1.01$ kg per orbit, $33.24$ kg for the whole nominal mission.
    An important note shall be made in regards of the planetary phase: propellant consumption is based on the 11-days orbit around Jupiter, which were never performed. Consumption during the real 53-days orbit is actually lower, as both gravity gradient and magnetic disturbances diminish with distance, allowing for a longer life mission.
\end{itemize}

A total of $142.24$ kg of fuel was found to be consumed by the AOCS. If margins are considered as in Homework 2, considering that no orbital corrections (OTMs and TCMs) were taken into account, the computed value is shy of the real on-board mass.