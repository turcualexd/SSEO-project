\section{Propulsion system architecture}
\label{sec:prop_architecture}

The spacecraft axes are defined as shown in \autoref{fig:axis}: the spacecraft is spinning along the +Z-axis, aligned with the HGA. The +X-axis is aligned with the MAG boom while the +Y-axis is in the direction of cross product between +Z-axis and +X-axis.

\cfig{axis}{0.75}{Axis description}

\subsection{Main Engine and RCS}
\label{subsec:me_rcs}

Juno is equipped with a dual-mode propulsion system: the bi-propellant ME uses the hypergolic couple hydrazine and nitrogen tetroxide ($N_2H_4 - N_2O_4$) while RCS uses hydrazine as monopropellant. This choice was made to simplify the design: fewer tanks are needed as fuel ones are shared between the two systems. Moreover, the choice of this specific hypergolic couple is dictated by the storage requirements of the mission: in a five-year cruise, reliability and sturdiness of the propulsive system were among the main drivers of the mission. Electric thrusters were discarded as TRL-9 technologies were required: other limitations such as power budget, an highly radioactive environment, weight and space inside the spacecraft required a more simple and light solution. 

The ME is a Leros 1b, built by Nammo \cite{Leros}, which produces about 662 N of thrust with $I_{s,me} = 318.6$ s. This particular engine is certified for a 42 minutes continuous burn and has a cumulative lifetime of 342 minutes, so the manoeuvres shown in \autoref{table:deltav} are compliant with this constraint. This engine is utilized during the DSMs, JOI and PRM. It is mounted inside the body of the spacecraft along the -Z-axis, centred between the propellant tanks and under the electronic vault. This solution has probably been adopted due to space requirements inside the Atlas V fairing and safety precautions during the cruise phase. The ME is also shielded by a hatch that opens when a manoeuvre is needed.
The RCS is used for TCMs, attitude control and general SK. The catalyst used to decompose the hydrazine is the S-405, based on iridium and aluminum\cite{s405}.
The whole RCS is composed by four REMs, each of them mounted along the $\pm$Y-axis on a pylon, as shown in \autoref{fig:spin2}. Every pylon houses three thrusters, the MR-111C by Aerojet Rocketdyne\cite{RCS_info}, pointed in different directions, each providing a thrust of 4.5 N with $I_{s,rcs} = 220$ s. Electrical power is required to operate the feeding valves, heating valves and the catalytic bed, amounting to a maximum of around 13 W\cite{RCS_values}. 

\cfig{spin2}{0.77}{Forward and Aft deck view}

%\rfig{architettura_prop}{0.55}{Propulsion system architecture}

\rfig{foto_thruster}{0.55}{RCS Forward mount}
The pylons are raised respectively by 74 cm on the forward deck and about 26 cm on the aft deck as shown in \autoref{fig:foto_thruster}.
 With the need of limiting even more the interaction of the exhaust gasses with the on board instruments, the HGA and solar panels, axial thrusters are canted 10° away from the Z-axis while the lateral thrusters are canted 5° away from the X-axis and 12.5° toward the Z-axis \cite{juno_inner}. This particular configuration of RCS is required since no RW nor any other type of active attitude control is present on the spacecraft: the ability of decoupling the forces and the momentum was thus needed.
Lateral thrusters are denominated with letter "L" while axial thrusters are denominated with letter "A", as can be seen both from \autoref{fig:spin2} and \autoref{fig:foto_thruster}. This configuration increased the overall sturdiness of the propulsion system as "A" thrusters could be used as replacement of the ME for small manoeuvres. A simplified scheme of the propulsion system has been developed in \autoref{fig:architettura_prop}.

\subsection{Manoeuvre Implementation Modes}
\label{subsec:manoeuvre_implemementation_modes}

Juno's manoeuvres can be performed in two different modes: \textit{vector-mode} and \textit{turn-burn-turn}.

The \textit{vector-mode} consists of separated and coordinated axial and lateral burns from RCS thrusters. As seen in \autoref{subsec:me_rcs}, the thrusters are not exactly perpendicular one to the other, so during a \textit{vector-mode} manoeuvre an induced axial $\Delta V$ is generated and must be compensated.

The \textit{turn-burn-turn} mode consists in a sequence of RCS and ME burns: first the spacecraft slews to the design spinning rate of 5 RPM, then the ME is ignited and the manoeuvre is performed. At last, the spacecraft slews back to its nominal spinning rate. This mode is used during all the ME burns. In this kind of manoeuvre the RCS uses the "L" thrusters on the REMs. 

% \cfig{foto_thruster}{0.45}{RCS fordward mount}
\pagebreak
\subsection{Tanks} 
\label{subsec:tanks}
%\rfig{foto_thruster}{0.5}{RCS fordward mount}
\rfig{architettura_prop}{0.55}{Propulsion system architecture}

Juno's tanks are equally distributed throughout its hexagonal shaped body. Four tanks are needed to store hydrazine and two tanks are needed to store the oxidizer. As can be seen from \autoref{fig:foto_thruster}, the oxidizer ones (green tanks) are located along the X-axis, while fuel ones (blue ones) are placed in the remaining bays. All six tanks work at 2.15 MPa, estimated as the sum of the nominal operational pressure of the ME and a small amount, 50 kPa, induced by the pressure losses of the system\cite{Leros}. On board tanks have a sphere-like shape: it allows to have the most internal volume with the lowest possible surface and so both weight and heat exchange are limited.

The two tanks\cite{2tankshe} containing the supercritical helium needed to pressurize the propellant system are assumed to be initially pressurized at 21.5 MPa, a value ten times higher than the nominal working pressure, and are placed near solar wing one and solar wing two. Unlike fuel and oxidizer tanks, helium ones do not have a sphere-like shape due to volume management inside the bays\cite{he_tank}: cylindrical tanks allow to fill better the gaps present under the main tanks. The positioning of the pressurizer tanks breaks even more the symmetry of the mass distribution: this feature, in concomitance with the distribution of the propellant tanks, will make the COM shift not only along the Z-axis unless other precautions were made in placing other internal components. 

A system of valves regulates the pressure inside the tanks to allow nominal operations of the ME and RCS. All the tanks are insulated from their surroundings and heating elements are present on both tanks and feeding lines to ensure safe and nominal temperature inlet for the whole propulsion system.
One of the main problem with managing liquid in space is the need of guiding the propellants to the feeding lines of the engines to avoid mixture of gas and liquids during the injection in the combustion chamber, compromising the correct functioning of the propulsion system. Moreover, liquid propellants produce sloshing movements that could apply forces and moments inside the tanks, causing an unsteady oscillatory spin. Juno is a spin-stabilized spacecraft so the induced forces would cause a movement of nutation. A Propellant Expulsion Device (PED) is thus needed: the spinning of the spacecraft helps guiding the propellant to the most exterior part of the tanks were feeding lines are located\cite{slosh}.
MEFs are tasks needed to flush the main propellant line in order to test the system after a long period of rest.
In order to accomplish its mission, Juno holds about $2000$ kg of propellant: about $1280$ kg of fuel and $720$ kg of oxidizer \cite{juno_inner}. A more detailed analysis of tanks will be conducted in \autoref{subsec:fuel_ox_sizing}.