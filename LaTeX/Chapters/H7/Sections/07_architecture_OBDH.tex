\subsection{Architecture of OBDH}
\label{subsec:OBDH_architecture}

The OBDH system is based on two redundant, single fault redundant C\&DH boxes, each including:
\begin{itemize}
    \item \textbf{RAD750 Processor:} a 3U radiation hardened single-board computer by BAE systems \cite{C&DH_power} with 256 MBytes of NVM flash memory and 128 MBytes of DRAM local memory. \cite{juno_sito} It's able to handle 100 Mbps of instrument throughput, much higher than needed for payloads requirements, and can operate at up to 200 MHz with a substantial performance improvement over older rad-hard processors. Furthermore the CPU itself can withstand a total radiation dose of up to 1 Mrad (Si) and has already been employed on various missions such as NASA's Mars Science Laboratory proving its effectiveness. \cite{RAD750} \cite{batterie}
    \item \textbf{DTCI card:} it contains the interface between the C\&DH box and all the instruments of the spacecraft, while also providing science data storage capabilities. In particular 32 Gbits are available for data storage with a further 8 Gbits dedicated to EDAC. This is sufficient both for all science orbit downlink data requirements and representative stress cases. Unlike all other cards present in the C\&DH box it's characterized by a 6U format instead of a 3U one. \cite{juno_sito}
    \item \textbf{GIF card:} similarly to the DTCI card it provides the interface between the C\&DH boxes and all GN\&C hardware, namely the various attitude sensors and the twelve RCS thrusters. As maintaining proper functionality of the AOCS is of critical importance two GIF cards are present in each box, thus offering quadruple redundancy. Each card is also connected to both SDSTs trough MIL-STD-1553 \cite{MIL-STD-1553} busses. \cite{juno_telecommunication}   
    \item \textbf{ULDL card:} it also connects to the two SDSTs and, as the name implies, it's tasked with providing and controlling both the uplink (command) and downlink (telemetry and data) of the S/C. In this case only one card is installed in each box, but, similarly to the GIF cards, both C\&DH units are cross-strapped to both SDSTs. This connection, though, utilizes LVDS interfaces instead of MIL-STD-1553 busses. \cite{juno_telecommunication}  
    \item \textbf{cPCI bus:} an internal bus that interconnects all of the C\&DH box hardware. 
\end{itemize}
The two C\&DH boxes are further connected to the rest of the spacecraft with two sets of redundant RS-422 busses: a synchronous interface for science data and an asynchronous interface for telemetry and command, both capable of a transfer rate of 57.6 Kbps. \cite{UVS_info} \cite{MWR_info} 
All P/Ls also feature individual computation capabilities and, in some cases, a certain level of local memory, with the only exception being the Gravity Science investigation since it relies only on the TMTC system. This architecture choice was selected as both the high number of P/Ls and their complexity meant that a centralized processor couldn't handle the required workload. 