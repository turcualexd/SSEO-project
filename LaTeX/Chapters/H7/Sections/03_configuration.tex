\section{Configuration inside the launcher}
\label{sec:config_launcher}

As mentioned in \autoref{sec:config_introduction}, the cross section of Juno's main body is an hexagon with 1.77 m sides. The circumscribed circumference has thus a diameter of 3.54 m. The solar arrays have a total area of about 60 m$^2$ and their total aperture when deployed is more than 20 m.\cite{masses_ref} Given that the needed launcher, the Atlas V, has as an option a fairing of diameters up to 4.57 m, the three arrays had to be folded. A system of hinges and struts and a division in multiple panels had to be developed.\cite{solar_panels_coef} Dimensions of the arrays led to difficult on ground testing: technicians had to fold and unfold the three arrays one by one and not all three together.\cite{solar_panels_testing} The adopted fairing features a maximum height of 10.18 m, that is the smallest fairing among the 5 m family\cite{atlas_manual}. Dimensions are coherent with the ones of Juno. The smallest possible fairing to be mounted on the Atlas V had an internal available diameter of 3.75 m 
\rfigII{juno_packed}{Juno in packed configuration}{0.3}{20}{-5mm}{-5mm}
and an available internal height of 1.58 m which was not compliant with dimensions of the S/C. 
The packed configuration features, in addition to the folded arrays, the Waves antennas to be retracted: together with the solar arrays they will be deployed soon after separation from the Centaur upper stage. 
% immagine di juno ripiegato 
The adapter used to connect the launcher to the spacecraft was found to be the type D1666, composed by two pieces of machined aluminum. The capabilities of the adapter vary depending on the position of the CG of the spacecraft with respect to the separation plane. Due to the mass distribution of Juno at launch, the CG was found to be inside the main body, at around 1.4 m from the said plane. From the Atlas V user manual\cite{atlas_manual}, this values imply a maximum payload capability of 6.5 tons, above the $\approx 3.62$ tons of Juno at launch.\cite{masses_ref}
In \autoref{fig:juno_packed}\cite{foto_fairing} it is possible to observe the S/C with the folded arrays and the adapter prior to the launch. Attention must be paid to A1, where the MAG is mounted. 
Separation of Juno from the adapter is carried out by a system of springs, a clamp band and a release mechanism: the system allows to safely separate the S/C and provides the needed energy to obtain positive separation, which in the case of Juno implies reaching an orbit with a speficic energy of 31.1 km$^2$/s$^2$. 

% come facciamo il deployment dei pannelli
% come stacchiamo juno dall'adattatore  Payload Separation System - inizio spin 
