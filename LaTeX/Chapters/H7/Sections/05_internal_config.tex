\subsection{Internal configuration}
\label{subsec:internal_config}

Only considerations about the internal structure and the Propulsion subsystem can be made, as all the important instruments are placed on the outside of the satellite or inside the vault.
A view of the internal configuration can be observed in \autoref{fig:nuovo_foto_interno}. Six propellant tanks are located inside the main body, four for fuel and two for oxidizer, one in each bay.
This configuration allows to improve the stability of the spinning stabilized platform and allows to have a better control on the position of the CG, that will shift along the Z-axis throughout the mission.
The tanks are all made of titanium to guarantee high specific strength and thermal insulation.\cite{LL_early_cruise}
Moreover, the tanks are covered with MLIs and electrically heated. Fuel's and oxidizer's tanks have a sphere-like shape, while the two remaining tanks containing helium are cylindrical with domed ends.
Propellant lines are located on the outer part of the tanks as while Juno spins, the fuel is naturally moved to the most exterior region.
The main engine is stored inside the main body, in the center region where the six bays converge. \cite{Juno_launch} \cite{Leros}
As the engine fires, heat must be dissipated to not overheat the nozzle. This constrain leads to the maximum operation of the engine of 42 minutes.\cite{Leros}
The same structure that is stressed during the launch is responsible for thermally decoupling the tanks and the ME's nozzle and, together with the thin panels that divide the bays, contributes to the torsional rigidity of the S/C.