\subsection{Introduction of Juno's configuration}
\label{subsec:config_introduction}

% Juno is the first spin-stabilized solar-powered spacecraft to conduct science operations around Jupiter, featuring different instruments with various requirements and limitations. Given the harsh environment it faces around the planet, the longevity of the mission and the required power to conduct operations, both packed and unpacked configuration of the payloads, antennas, internal hardware and power sources required special attention. Mass distribution is also a critical factor in the accuracy of the different measurement as Juno spins at 2 RPM during science operations. In the following paragraphs, first the general configuration of Juno will be carried out, than the analysis of the various payloads will be presented.
Juno is the first spin-stabilized, solar-powered spacecraft to perform scientific operations around Jupiter, equipped with various instruments each having unique requirements and limitations. Due to Jupiter's harsh environment, special consideration was needed for the configuration of payloads, antennas, internal hardware, and power sources to ensure the mission's longevity and operational power. Accurate mass distribution is critical for precise measurements, as Juno spins at 2 RPM during science operations.\cite{Juno_launch} The following sections will detail Juno's general configuration and analyze its various payloads.
The external appendages of Juno and its reference system is shown in \autoref{fig:juno_config}.