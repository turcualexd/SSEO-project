\subsection{Shape and appendages}
\label{subsec:shape_appendages}

\rfigII{juno_config}{Juno's configuration and reference system}{0.65}{13}{-5mm}{-6mm}

The main body of the S/C has the shape of a regular hexagonal prism with sides of 1.77 m and and high of 1.52 m: in its interior the ME, tanks of fuel and oxidizer are stored. The interior of the prism is divided into six bays thanks to honeycomb composite walls that allow, together with MLIs and thermal blanket, to decouple each section from their surroundings. A tank is placed in each bay. On the top deck are installed the two 55 Ah Li-Ion batteries, the two SRUs, the SSSes, all the cabling for the instruments, JEDI and JADE. Moreover two of the four REMs are placed on the top deck along the Y-axis, the titanium vault is placed on the center of the prism, aligned with the Z-axis. Inside the vault all the important electronics is stored in order to keep it shielded from the radiations. At last, the 2.5 m diameter HGA to transmit and receive in X-band and Ka-band is mounted on top of the vault. Alongside the HGA, the front LGA and the MGA are placed. On three of the six lateral faces of the main body the three solar arrays are mounted and spaced 120° apart. Under the solar array along the +X-axis the WAVES antennas are placed, while on a fourth face both the JunoCam and the UVS instruments are mounted. The last two lateral faces are occupied by the Microwaves Radiometer. The three solar arrays are composed by a different number of panels: A1 has three panels while A2 and A3 are composed by 4 panels each. In fact on the edge of A1 the MAG suites of instruments is mounted. The aft deck, aligned with the -Z-axis, features the ME cover, the last two REMs for the ADCS, the aft LGA, the TLGA and JIRAM.
