\section{Reverse sizing of EPS}
\label{sec:EPS_sizing}

As discussed in \autoref{sec:EPS_phases}, the most demanding phases are the GRAV science perijoves, which have a duration of about 6 hours each. In particular, the most critical one occurs when Jupiter is at its aphelion.
This specific point is the one chosen for the sizing of the EPS.
Firstly, the dimensioning of the solar panels will be carried out without considering the presence of the secondary power source.
This is done to overestimate the required surface at Jupiter in order to satisfy the whole power requirement.
Later, the real solar panels are assumed and the batteries are sized in order to fill the actual gap between the primary energy source and the power required by the whole system. 


\subsection{Solar panels}
\label{subsec:solar_panels_sizing}

To compute the solar flux incident on the panels at the design orbit point, the following equation has been adopted:

\begin{equation}
    q_{sun} = \frac{q_0}{D^2} \cos \theta \quad [ \textrm{W/m}^2 ]
\end{equation}

where $D = 5.4543$ AU is the distance of the S/C from the Sun, $q_0$ is the solar flux at the distance of $1$ AU from the Sun and $\theta$ is the angle between the Sun direction and the normal from the panel surface.
Since during GRAV the S/C is Earth pointing, $\theta$ coincides with the SPE angle, which can be found from ephemeris ($\approx 7.48$°).

From $q_{sun}$ and the power required in this condition ($P_{req} = 400$ W from \mref), the total area required to satisfy the power demand at perijove is computed as:

\begin{equation}
    A_{sa}' = \frac{1.2 \, P_{req}}{q_{sun} \, \varepsilon \, \left( 1 - dpy \right) ^ {yrs} \, I_D} = 62.85 \; \textrm{m}^2
    \label{eq:A_sa}
\end{equation}

\autoref{eq:A_sa} takes into account the degradation of the panels during the mission ($yrs = 5.79$ years is the time elapsed from the launch) and applies a $20\%$ margin on the power demanded \mref.
Typical values for GaAs UTJ panels are assumed and reported in \autoref{table:panels_values}:

% perchè 5.79 years? 
% cosa vuol dire " it is worth noting that A′sa does not take into account the discrete distribution of areas due to cells."?
%  a cosa si riferiscono le cose con ' ? 
% 

\begin{table}[H]
    \renewcommand{\arraystretch}{1.3}
    \centering
    \begin{tabular}{|c|c|c|}
        \hline
        $\boldsymbol{\varepsilon}$ \textbf{[-]}    &
        $\boldsymbol{dpy}$ \textbf{[-]}     &
        $\boldsymbol{I_D}$ \textbf{[-]}     \\
        \hline
        \hline
        0.3 & 0.0375 & 0.77 \\
        \hline
    \end{tabular}
    \caption{Properties assumed for solar arrays \mref}
    \label{table:panels_values}
\end{table}

It is worth noting that $A_{sa}'$ does not take into account the discrete distribution of areas due to cells.
Moreover, an additional string of cells must be added to satisfy the official margin by ESA \mref.
A more refined calculation is shown in \autoref{eq:A_sa_ref}.

\begin{equation}
    n_{cells}' = \left\lceil \frac{A_{sa}'}{A_{cell}} \right\rceil \qquad
    n_{series} = \left\lceil \frac{V_{nom}}{V_{cell}} \right\rceil \qquad
    n_{cells} = \left\lceil \frac{n_{cells}'}{n_{series}} \right\rceil
                \cdot \left( n_{series} + 1 \right) \qquad
    A_{sa} = n_{cells} \cdot A_{cell}
    \label{eq:A_sa_ref}
\end{equation}

The area of a single cell is taken from the technical sheet \mref: $A_{cell} = 26.6 \; \textrm{cm}^2$.
The voltage of each cell at Jupiter is taken from a model\cite{solar_panels_coef} that takes into account the distance from the Sun and the low operative temperature and no difference have been considered between cells belonging to different strings. The used value is thus $V_{cell} = 2.77$ V.

The solar arrays have a complex distribution of cells in three different types of series, hence with different voltages (as already discussed in \mref). % -- non ho capito questa frase ---
To keep the calculation simpler, an average on the number of cells in the series has been computed through the nominal voltage of the system ($V_{nom} = 28$ V).
The results of computation are compared to the real arrays in \autoref{table:panels_results}.

\begin{table}[H]
    \renewcommand{\arraystretch}{1.3}
    \centering
    \begin{tabular}{|c|c|c|}
        \hline
        & $\boldsymbol{n_{cells}}$ \textbf{[-]} &
        $\boldsymbol{A_{sa}}$ \textbf{[$\boldsymbol{\textbf{m}^2}$]} \\
        \hline
        \textbf{Sizing results} & 25788 & 68.60 \\
        \hline
        \textbf{Real values}\cite{masses_ref}\mref & 18698 & 49.74 \\
        \hline
    \end{tabular}
    \caption{Results and comparison of the solar arrays}
    \label{table:panels_results}
\end{table}

As can be seen in \autoref{table:panels_results}, the sized arrays result to be noticeably larger with respect to the real panels.
This is due to the fact that this phase utilizes both the primary and the secondary sources and the batteries are not considered in this first preliminary sizing.

\subsection{Batteries}
\label{subsec:battery_sizing}

For the battery sizing, the real active area of solar arrays was assumed (\autoref{table:panels_results}).
From this, the power required from the battery was computed as the difference between the required power (from \mref) and the one delivered by the solar panels: % --- che required power ? quando ? ----

\begin{equation}
    P_{req} = 530.16 \; \textrm{W} \qquad
    P_{sa} = q_{sun} \, \varepsilon \, \left( 1 - dpy \right) ^ {yrs} \, I_D \, A_{sa}^{real} = 419.52 \; \textrm{W} \qquad
    P_{bat} = P_{req} - P_{sa} = 110.64 \; \textrm{W}
\end{equation}

This power has to be delivered by the batteries in proximity of the perijove for approximately $T_{pj} = 6$ h. From \mref it is possible to obtain the capacity required by the battery in order to satisfy this request:

\begin{equation}
    C = \frac{T_{pj} \, P_{bat}}{\eta \, DoD \, V_{nom}} = 49.9132 \; \textrm{Ah}
\end{equation}

where the line efficiency $\eta$ is assumed to be $95\%$ and the $DoD$ is assumed $50\%$ to be conservative and to not excessively reduce the battery life cycles along the mission. The result is compliant with the chosen battery, which capacity is $55$ Ah 
\mref.
Moreover, since the battery is a 6s1p type, hence has only one series of cells, an additional battery is added to the system for cold redundancy as requested from ESA margins 
\mref. 
Subsequently, a series of calculations were conducted to verify the capacity of the solar panels to recharge the batteries in the orbital region where scientific operations are not conducted. The time available to recharge the batteries was computed as the difference between the nominal orbital period of 11 days and the 6 hours period passed at the perijove while performing science operations. The time necessary to recharge the batteries is then calculated with \autoref{eq:time_charging}
\begin{equation}
    \label{eq:time_charging}
    t_{av} = T - 6 \textrm{h} = 258 \, h \qquad
    t_{ch}=\frac{C \, DoD \, V_{nom}}{P_{ch}}=17.47 \,h
\end{equation}
where $P_ch$ is the value shown in \autoref{sec:EPS_phases}. The solar panels are capable of recharging the batteries since $P_{sa}=419.52 \, W$ is higher than the power requested in BTT mode 
\mref 
and $t_{ch}$ is widely less than $t_{av}$. 

Obtained results are compliant with the real mission.  
% Aggiungere parte di ricarica batteria

% --- non abbiamo mai detto che la richiesta di potenza durante l'eclissi non è un problema. è vero che la grav è la condizione più gravosa ma se non esplicitiamo ogni cosa non capiscono 


