\section{Architecture and rationale of EPS}
\label{sec:EPS_architecture_rationale}

\subsection{Available alternatives}
\label{subsec:available_alternatives}

The endeavour that Juno faces to generate enough electricity to sustain science operations at 5.44 AU required particular attention in designing an efficient and reliable electric control system. Particularly, different options were present to generate the amount of power required, each of them with advantages and disadvantages:

\begin{itemize}
    \item \textbf{RTG:} the choice of a radioisotope to generate electricity could be considered. In order to generate the amount of power required around Jupiter, during the planetary phase 
    (\mref), 
    one RTG of the same size mounted on Kuiper would have been sufficient. Electric power requirement would also be lower as some heat could have been routed to the propulsion section to heat up the tanks. This choice however had some problems, mainly with respect to the Earth EGA, radiation contamination, heat dissipation and weight distribution. Availability of Plutonium-239 was also critical as suppliers could not guarantee the needed amount of fuel.
    \mref % ---- eventualmente da approfondire ----- 
    \item \textbf{Solar panels:} no spacecraft equipped with solar panels has ever been tested at 5.44 AU from the Sun. This choice would have required a very large surface area, with the need of precautions inside the fairing during launch operations, to provide enough power for safe operations at Jupiter and a complex management system in order to not discharge too much current inside the electronics during the ICs and the OC.
\end{itemize}

Considering also the driving requirement of utilizing as much as possible off the shelf component, the founding constrain and the different possible configuration offered, solar panel were chosen. 

\subsection{Components and distribution}
\label{subsec:components_and_distribution}