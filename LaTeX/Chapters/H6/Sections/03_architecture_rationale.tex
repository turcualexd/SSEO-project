\section{Architecture and rationale of EPS}
\label{sec:EPS_architecture_rationale}

The endeavour that Juno faces to generate enough electricity to sustain science operations at 5.44 AU required particular attention in designing an efficient and reliable electric control system. Particularly, different options were present to generate the amount of power required, each of them with advantages and disadvantages:

\begin{itemize}
    \item \textbf{RTG:} the choice of a radioisotope to generate electricity could be considered. In order to generate the amount of power required around Jupiter, during the planetary phase (\mref), one RTG of the same size mounted on Kuiper would have been sufficient. Electric power requirement would also be lower as some heat could have been routed to the propulsion section to heat up the tanks. This choice however had some problems, mainly with respect to the Earth EGA, radiation contamination and weight distribution. 
\end{itemize}