\section{Architecture and rationale of TCS}
\label{sec:TCS_architecture_rationale}

The TCS of Juno must tackle a wide range of thermal environments, as discussed in \mref.
The cold case however is the most critical condition for the S/C, so TCS is mainly designed on this situation: the vault, the main body and the external hardware are all thermally insulated and decoupled. Heaters are also present on each individual section/sensor. This guarantees flexibility in order to ensure the operating temperature for each component. Despite the insulation, more than half of the power generated at Jupiter is demanded by the TCS just to heat up the S/C. 

Four main zones were identified for the following analysis:

\begin{itemize}
    \item \textbf{Vault}:  all the main electronic hardware is contained here. The size of this box is 0.8 m $\times$ 0.8 m $\times$ 0.7 m. The lower surface is attached to the main body while the top surface is linked to the HGA, lateral surfaces points outwards and mainly to deep space. The walls are made of 1 cm thick titanium walls. This metal has a low conductivity value ($\approx$ 6.7$\div$7.4 W/mK) which is a positive feature for the cold case at Jupiter. Also, major heat generation happens during science orbits since all the instrument electronics is powered on. In addition, the low external thermal flux during science imposes additional requirements in relation to the optical properties of the lateral vault surfaces. 
    Tantalum MLI blankets were used in order to ensure both low emissivity and absorptivity ($\epsilon \approx 0.01 \div 0.035$ \mref). 
    However, during the phases in which the thermal flux is at its highest (TP3), the vault shall be able to dissipate enough power. This is in contrast with the above mentioned design choices. To ensure compatible thermal environment in the vault, three louvers were applied on its external lateral surfaces in order to point deep space and have an efficient IR emission. The dimension of a single louvre is 0.53 m $\times$ 0.40 m, two of them are placed vertically while one of them is placed horizontally. The motivation for this choice is relative to the internal configuration of the electronics.
    The opening of the louvre' shutters raises the emissivity value from 0.14 to 0.74 (\mref produttore), enabling higher out-going radiative heat flux. 
    The justification for this passive and low complexity solution  was mainly due to the fact that the hot case scenario was encountered only during a restricted time of the overall mission. Moreover, the louvre technology effectiveness was tested and ensured by previous interplanetary mission such as Rosetta and New Horizons. 
    However, most of the radiation coming from the sun was shielded by the HGA which protected the vault. 
    The Germanium coated Kapton used to cover the antenna dish has an operating range temperature of -200°C $\div$ +200°C, while its absorptivity and emissivity values are $\alpha = 0.568$ and  $\epsilon = 0.72$ respectively\mref.
    The HGA will heat up and exchange radiative heat with the lower vault, hence the necessity to dissipate heat also from the electronic vault with louvres.
    The electronics contained in the vault are tightly packed to reduce the effects of internal reflection of high energy particles that can still penetrate the walls. From a thermal viewpoint this means that the generated heat is better retained and the internal temperature of the hardware is fairly uniform.
    
    \item \textbf{Main body:} it is the hexagonal prism that contains most of the propulsion subsystem hardware (propellant and pressurizer tanks, feeding lines and ME). Two payload sensors are also present inside, namely UVS and JunoCam. All of these elements require separated strategies to manage the temperatures. 
    
    The six propellant spherical tanks (two of oxidizer and four of fuel) are arranged into six bays that corresponds to the equally distributed volumes of the the hexagonal prism. Hence, each compartment contains just one tank and it is thermally uncoupled from the others in order to guarantee a better independent and redundant thermal management. To ensure this uncoupling, high reflectance blankets are used over the tank surface. Nominally, aluminized polyester film is used ($\alpha / \epsilon \approx$ 3.5) this material minimizes heat flow to and from the S/C, it is generally used for temperature ranges from -250°C to +120°C and has been successfully used on previous mission (link bibliografia). The tanks are made of titanium which has low thermal conductivity. 
    The honeycomb composite lateral walls of each bay are also covered with high reflectance coating to ensure radiative insulation.
    In addition, heaters are present into the propellant tanks, helium tanks (reference paolo) and also feeding lines. \mref
    Other thermal considerations on the internal main body refers to the  operations of the main engine which is mainly inside the central body. In that moments high thermal flux must be handled by the internal structure which must be thermally decoupled both radiatively and conductively. (manca materiale facce laterali main body)
    \item \textbf{External hardware:} which comprehend all the remaining payload sensors ()
    \item \textbf{Solar panels:} 
\end{itemize}


