\section{Architecture and rationale of TCS}
\label{sec:TCS_architecture_rationale}

The TCS of Juno must tackle a wide range of thermal environments, as discussed in \mref.
The cold case however is the most critical condition for the S/C, so TCS is mainly designed on this situation: the vault, the main body and the external hardware are all thermally insulated and decoupled. Heaters are also present on each individual section/sensor. This guarantees flexibility in order to ensure the operating temperature for each component. Despite the insulation, more than half of the power generated at Jupiter is demanded by the TCS just to heat up the S/C.

\cfig{render_juno}{0.9}{Thermal subdivision of Juno}

\vspace*{-3mm}

Four main zones were identified for the following analysis:

\begin{itemize}
    \item \textbf{Vault}:  all the main electronic hardware is contained here. The size of this box is 0.8 m $\times$ 0.8 m $\times$ 0.7 m. The lower surface is attached to the main body while the top surface is linked to the HGA, lateral surfaces points outwards and mainly to deep space. The walls are made of 1 cm thick titanium walls. This metal has a low conductivity value ($\approx$ 6.7$\div$7.4 W/mK) which is a positive feature for the cold case at Jupiter. Also, major heat generation happens during science orbits since all the instrument electronics is powered on. In addition, the low external thermal flux during science imposes additional requirements in relation to the optical properties of the lateral vault surfaces. 
    Tantalum MLI blankets were used in order to ensure both low emissivity and absorptivity ($\epsilon \approx 0.01 \div 0.035$ \mref). 
    However, during the phases in which the thermal flux is at its highest (TP3), the vault shall be able to dissipate enough power. This is in contrast with the above mentioned design choices. To ensure compatible thermal environment in the vault, three louvers were applied on its external lateral surfaces in order to point deep space and have an efficient IR emission. The dimension of a single louvre is 0.53 m $\times$ 0.40 m, two of them are placed vertically while one of them is placed horizontally. The motivation for this choice is relative to the internal configuration of the electronics.
    The opening of the louvre' shutters raises the emissivity value from 0.14 to 0.74 (\mref produttore), enabling higher out-going radiative heat flux. 
    The justification for this passive and low complexity solution  was mainly due to the fact that the hot case scenario was encountered only during a restricted time of the overall mission. Moreover, the louvre technology effectiveness was tested and ensured by previous interplanetary mission such as Rosetta and New Horizons. 
    However, most of the radiation coming from the sun was shielded by the HGA which protected the vault. 
    The Germanium coated Kapton used to cover the antenna dish has an operating range temperature of -200°C $\div$ +200°C, while its absorptivity and emissivity values are $\alpha = 0.568$ and  $\epsilon = 0.72$ respectively\mref.
    The HGA will heat up and exchange radiative heat with the lower vault, hence the necessity to dissipate heat also from the electronic vault with louvres.
    The electronics contained in the vault are tightly packed to reduce the effects of internal reflection of high energy particles that can still penetrate the walls. From a thermal viewpoint this means that the generated heat is better retained and the internal temperature of the hardware is fairly uniform.
    
    \item \textbf{Main body:} it is the hexagonal prism that contains most of the propulsion subsystem hardware (propellant and pressurizer tanks, feeding lines and ME). Two payload sensors are also present inside, namely UVS and JunoCam. All of these elements require separated strategies to manage the temperatures. 
    
    The six propellant spherical tanks (two of oxidizer and four of fuel) are arranged into six bays that corresponds to the equally distributed volumes of the the hexagonal prism. Hence, each compartment contains just one tank and it is thermally uncoupled from the others in order to guarantee a better independent and redundant thermal management. To ensure this uncoupling, high reflectance blankets are used over the tank surface. Nominally, aluminized polyester film is used ($\alpha / \epsilon \approx$ 3.5) this material minimizes heat flow to and from the S/C, it is generally used for temperature ranges from -250°C to +120°C and has been successfully used on previous mission (link bibliografia). The tanks are made of titanium which has low thermal conductivity. 
    The honeycomb composite lateral walls of each bay are also covered with high reflectance coating to ensure radiative insulation.
    In addition, heaters are present into the propellant tanks, helium tanks (reference paolo) and also feeding lines. \mref
    Other thermal considerations on the internal main body refers to the  operations of the main engine which is mainly inside the central body. In that moments high thermal flux must be handled by the internal structure which must be thermally decoupled both radiatively and conductively. (manca materiale facce laterali main body)
    \item \textbf{External hardware:} it comprehends all the remaining payload sensors (JADE, JEDI, Waves, MWR, MAG, JIRAM), SRUs, SSSes, batteries, antennas and RCS. Waves and JIRAM are located on the aft deck which can be thought as a cold side since it points deep space. JIRAM does not have a minimum temperature range for its sensor, the maximum operating temperature is 95K in order to have useful scientific data. Waves sensing unit is composed of the antenna and the pre-amplifier unit which is thermally coated. Conversely to the sensors on the aft deck, JADE and JEDI have minimum operative temperature so they are located on the surface that is in visibility of the Sun (top deck). Moreover, the 4 sensors of JADE and the sensor of JEDI are not directly shielded by the HGA. MWR antennas are positioned on two of the six lateral surfaces of the main body. No stringent requirements are presented for this hardware. During Juno operative life at Jupiter these sensors will face the planet's surface, hence the incoming radiation is from the albedo and infrared. 
    MAG payload positioning is much more critical under the thermal point of view. The payload is made of two MOBs, each one of them contains the FGM and two CHUs. The two boards are for redundancy. 
    For mechanical and thermal stability requirements, the two MOBs are made of Carbon Silicon Carbide which in turn are linked to the MAG boom made of aluminum honeycomb between two carbon sheets. To ensure thermal decoupling from the structure, the joints between the magnetic boom and MOBs are made of titanium. 
    Regarding the sensors, both the FGMs and CHUs are located on the side that is pointing deep space.
    The FGM sensor is enclosed individually within a multilayer thermal blanket and thermally stabilized at all times by a non-magnetic resistive heater driven by an alternating current, which keep the FGM operating temperature close to 0°C. The temperature gradient between the FGM unit and the MOB is of 80°C, to reduce the thermal strains the joints between these two units are made of three titanium joints.  This has a double effect: reduce the rotation induced by the gradient and reduce the thermal coupling due to the low conductivity and smaller area of contact.
    Conversely, the CHUs run at much lower temperature (-54°C) and thermal gradient is much lower. Their body is made of titanium, they are thermally conditioned with resistive heaters and enclosed into a MLI blanket\mref (MAG)

    The SRUs (two for redundancy) are located on the top deck, so they receive solar flux while S/C is pointing Earth or Sun. Their hardware is sensible to radiation so the internal components are heavily shielded with tungsten and titanium, the first is thermally conductive while the second is not. 
    Moreover, copper is used as a heat sink to transport thermal energy on the CCD, which is the sensible part of the apparatus. \mref

    RCS contains also thermal sensible materials, such as the catalytic bed Shell-405 which cannot withstand very cold environment and it is susceptible to thermal cycles. As a consequence, each set of the three thrusters is heavily shielded and heaters are placed to maintain a minimum temperature.

    Most of the external hardware is covered with MLI which guarantees radiative decoupling with low values of emissivity and absorptivity.Moreover, these layers serve as additional protection to the high energy particles radiation of the Jupiter environment.

    \item \textbf{Solar panels:} three solar arrays are present on the spacecraft to provide electricity throughout the different phases of the mission. They are connected to the main body through hinges and struts. To guarantee the correct pointing towards the Sun, stiffness was required to avoid deformation during the spin. The configuration allows each panel to have a clear view of both Sun and DS: dissipation of heat occurs from both front and back faces as the view factor between panels and Juno's main body is close to 0. 
    To ensure the proper functioning in drastically diverse environments, solar panels were designed to withstand a wide range of temperatures, going from the predicted +100 °C of the perihelium to the -140 °C around Jupiter\cite{solar_panels_coef}. Moreover, the whole spacecraft has to survive to radiations up to $10^{15}$ MeV so a custom made CMG coverglass with coating was employed. Every cell of the panels is thus covered on the front side with a 304 $\mu$m, 14.3 mil fused silica equivalent\cite{solar_panels_coef}, anti-reflective coating and Indium Tin Oxide, the first to improve performance in a LILT environment, the second to mitigate surface charge buildup\cite{solar_panels_coef}. The rear side shielding of each panel's substrate is shielded with a 30 mils fused silica equivalent of Kapton and Germanium. In order to guarantee the correct amount of electricity during science operations, around 440 W at JOI and 400 W EOM, the arrays are drastically oversized at any distance below 5.4 AU: at 1 AU 14 KW are produced. During the ICs and OC, more electricity than needed is generated but left on the panels in the form of heat that must be dissipated from both front and back faces. To do so, different kinds of coatings are used: % values of absorptivity vary from 0.7 to 0.92 for Kapton, placed on the front side, while 
    values for emissivity vary from 0.84 to 0.88 for Black Kapton and from 0.66 to 0.88 for the Kapton on the back side. Coverglass on the front side is transparent to the incoming radiation, but it is able to emit towards the DS with an emissivity ranging from 0.73 to 0.82. All the intervals for the reported coefficients are related to hot and cold case\cite{solar_panels_coef}.
\end{itemize}