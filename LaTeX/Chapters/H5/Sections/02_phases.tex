\section{Analysis of thermal conditions along the mission}
\label{sec:TCS_phases}

\subsection{Thermal phases analysis}
\label{subsec: TP_analysis}

Different thermal conditions have been encountered by Juno during its cruise. In previous chapters, the mission was divided into phases by different attitude and communication constraints. These phases will be now grouped by the means of thermal constraints to better analyze their evolution during the mission time.

\begin{itemize}
    \item \textbf{TP-1}:
    in this first phase, which comprehends both LEOP and IC-1, the S/C is in SPM due to thermal and power requirements. In particular, since the trajectory is relatively close to the Sun, Juno has to protect the vault with the HGA (as already explained in the previous chapters).
    % It is a hot phase but not the most critical one and so it will not be considered in the following analysis.
    Even if TP-1 is considered a hot phase, it is not the most critical as other phases present more stringent requirements, facing longer periods closer to external heat sources (i.e. Sun and Earth). 
    \item \textbf{TP-2}: 
    this second phase, among the ICs, is the longest, the only featuring EPM and does not call for any particular thermal requirement being Juno farther from both Sun and Earth, but closer than during the jovian phase. No specific attitude is required to thermally control the S/C  during the different manuevers performed during IC-2. Neither hot nor cold phase is considered along TP-2.    
    
    \item \textbf{TP-3}:
    %during the third thermal phase, consisting mainly of IC-3, Juno faces the hottest environment: Earth approach is executed in SPM to protect the electronics inside the vault as the S/C passes through the perihelion at 0.88 AU from the Sun. This condition was analyzed together with the passage of Juno inside Earth's SOI, where albedo and IR radiations had to be considered: these further radiations  
    the third thermal phase consists mainly of IC-3, performed in SPM to protect the electronics inside the vault as the S/C passes through the perihelion at 0.88 AU. This phase differs from IC-3 since the EGA belongs to a different TP and will be analyzed separately. During TP-3, Juno was found to face the most relevant hot environment, occurring at the closest approach to the Sun.  %--- QUI AVEVAMO MESSO HOTTEST MA NON è IL PIù CALDOOOO ALEX LEGGI -----------------------------------------
      
    \item \textbf{TP-4}: 
    the fourth phase analyzed consists only of the EGA, from the entrance till the exit of Juno from Earth's SOI. According to the model that will be described inside \autoref{subsec: heat_flux_analysis}, TP-4 faces the highest incoming thermal flux, 
    % scrivere heat flux due to....
    however this condition is only experienced for a few minutes, making it less relevant than the perihelion condition, which is hance the design point.    
    
    \item \textbf{TP-5}: 
    
    
    \item \textbf{TP-6}:
    
    
\end{itemize}

\subsection{Heat flux analysis}
\label{subsec: heat_flux_analysis}