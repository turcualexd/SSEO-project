\section{Introduction of TCS}
\label{sec:TCS_introduction}

The Thermal Control System of Juno adopts various strategies in order to maintain the instrumentation within operative ranges of temperature. This is done through both active and passive systems, which will be analyzed in \autoref{sec:TCS_architecture_rationale}.
First thing first, an analysis of the mission will be conducted to enlighten the thermal conditions the satellite is exposed to, which range from really hot environment nearby the Sun to extremely cold environment nearby Jupiter.
A selection of the two most extreme situations will be done through a preliminary evaluation of the heat fluxes in these phases.
In the light of this, the architecture of the Juno's TCS will be studied and justified through a brief rationale analysis.
Finally, a reverse sizing will be carried out imposing some simplifying assumptions in order to find the temperatures on Juno and to verify the compliance with its mission.

\begin{comment}
The Attitude and Orbital Control System of Juno comprehends various sensors and actuators that are vital to maintain the satellite in operability conditions and to execute all the basic tasks.
In this chapter, the main modes of the satellite are deduced through the analysis of the mission already done in previous chapters. These modes will be then arranged on the timeline consequently.
Based on the identified modes and pointing budget, the architecture of the actual system will be presented and then progressively analyzed, verifying the compliance with the previously found requirements.
Finally, a reverse sizing of AOCS will be carried out.
\end{comment}