\section{Mission analysis}
\label{sec:mission_analysis}

\subsection{Launch and cruise}
\label{sec: lancio e crociera}

The spacecraft was launched into orbit with an Atlas V 551 Rocket from Cape Canaveral. The actual launch date belonged to a 21-day time window limited by a number of events and their timings such as the Deep Space Maneuvers, the Earth Flyby, the Jupiter Insertion and the science orbits. % The adopted transfer strategy allowed for significant reduction in $Delta V$ with respect to a direct transfer between Earth and Jupiter.

Following the launch, after booster separation, Juno was put in a parking orbit around Earth thanks to a first burn of the Centaur upper stage. Afterwards, at time L+645 $s$, via a second burn given by the same stage, Juno entered an heliocentric trajectory. Solar arrays were deployed and initial checks on the instruments were performed at this time. % This procedure is fundamental in order to provide enough electrical power to the spacecraft to perform initial check on its health. 
The specific trajectory followed by Juno is called "2+ dV-EGA", which means that the spacecraft will perform an Earth gravity assist at around two years after launch.
During the initial cruise various correction maneuvers were performed: the main ones being the two DSMs needed to place Juno on the correct path to achieve the planned fly-by. 
DSMs were performed near the apocentre, located farther away from the Sun than Mars' orbit, causing the spacecraft to pass as close al 0.88 $AU$ before approaching Earth. % During the approach to perform the fly-by, attitude corrections were performed to protect the spacecraft by the incoming radiation from the Sun. 
The fly-by around Earth occurs on the $10^{th}$ of September 2013 and puts the spacecraft on its final trajectory to Jupiter. Particularly, the fly-by gave the spacecraft 7.3 $km/s$, avoiding a fire-up of the Leros 1b main engine of Juno. 


\subsection{Jupiter insertion}
\label{sec: joi}

\subsection{Science operations}
\label{sec: science ops}

\subsection{Extended mission}
\label{sec: extended}

\subsection{Mission disposal}
\label{sec: disposal}

