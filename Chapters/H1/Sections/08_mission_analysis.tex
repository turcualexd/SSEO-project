\section{Mission analysis}
\label{sec:mission_analysis}

\subsection{Launch and cruise}
\label{sec: lancio e crociera}

The spacecraft was launched into orbit with an Atlas V 551 from Cape Canaveral. After booster separation, Juno was put in a parking orbit around Earth thanks to a first burn of the Centaur upper stage. Afterwards, at time L+645 $s$, via a second burn given by the same stage, Juno entered an heliocentric trajectory. Solar arrays were deployed and initial checks on the instruments were performed at this time.


The specific trajectory followed by Juno is called "2+ dV-EGA", which means that the spacecraft will perform an Earth gravity assist at around two years after launch.



\subsection{Jupiter insertion}
\label{sec: joi}

\subsection{Science operations}
\label{sec: science ops}

\subsection{Extended mission}
\label{sec: extended}

\subsection{Mission disposal}
\label{sec: disposal}

