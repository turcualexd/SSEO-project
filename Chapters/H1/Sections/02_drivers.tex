\section{Mission drivers}
\label{sec:drivers}

Given the nature of the mission, being JUNO an interplanetary mission sarting from a distance of around 1 AU, with an nominal distance from the sun of 5.2 AU, and operating in an highly radiation 
intence envirorment, the following drivers have been identified: 

\begin{enumerate}[leftmargin=1.5em]
    \item \textbf{Provide enought electricity during the duration of the mission} 
    \newline The jurney of JUNO is long and passes through different zon of the solar system. Solar panels were chosen to provide 
    electic energy across the mission over a nuclear source, since it has been decided that it was better to advance tecnology
    of solar cells. The sistem needed is thus oversized at 1 AU with rispect to the science operations needed, the radiation on Jupiter is in fact up to 
    97 \% lower. Furthermore the scince operations are scheduled to begin around 5 years into the mission, so degradation of the solar cells 
    must be taken into account. INSERIRE RIFERIMENTI: VIDEO, SLIDES LAVAGNA, JUNO Mission
    \item \textbf{Provide the correct temperature range across the instruments during operation and the jurney}
    \newline The delicate suites of instrument present onboard JUNO require a very narrow range of temperature to operate, as 
    low as 1 $K$. The MWR instruments infact needs is to measure with hight accuracy the microwawe emission from Jupiter. During the cruise phase JUNO will
    pass as close as 0.88 AU from the Sun, so the S/C will need to protect the p/l from the incoming heat. Passive solutions are preferable to mantain semplicity 
    and reliability during the whole mission period. LINK NASA JPL
    \item \textbf{Shield the instruments from the radiation of Jupiter}
\end{enumerate}  