\section{Mission drivers}
\label{sec:drivers}

Given the nature of the mission, being JUNO an interplanetary mission sarting from a distance of around 1 AU, with an nominal distance from the Sun of 5.2 AU, and operating in an highly radiation 
intence envirorment, the following drivers have been identified: 

\begin{enumerate}[leftmargin=1.5em]
    \item \textbf{Providing enought electricity during the duration of the mission} 
    \newline The jurney of JUNO is long and passes through different regions of the solar system. Solar panels were chosen to provide electic energy across the mission over a nuclear source, since it has been decided that it was better to advance tecnology of solar cells rather than developing a new reactor. The system needed is thus oversized at 1 AU: the radiation on Jupiter is in fact up to 96\% lower on Jupiter than on Earth. Furthermore the operations are scheduled to begin around 5 years into the mission, so degradation of the solar cells must be taken into account. The final design consists in 3 solar arrays, 9 by 2.65 $m$ each, resulting in about 60 $m^2$ and granting a maximum power of 14 $KW$ around Earth and up to 500 $W$ around Jupiter. It was the first S/C to operate with solar pannel at such distance from the Sun. They are mounted on the side of exagonal body of the S/C at 120\textdegree one by the other and are deployed after booster separation. The spasecraft is spinning at around 1.4 $RPM$ during this phase and the deployment of the solar arrays slows it down. A crucial element must bu considered during the cruise phases: nominally the high gain antenna, that is mounted with the same positive direction of the solar arrays, is doing Earth pointing to communicate data but, at some point during the cruise phase, due to the proximity of the Sun, thermal requirments dictate a change in attitude and thus the spacecraft won't be pointing directly the Sun. Moreover, since the fly-by around Earth is done to gain $\Delta V$, the S/C will be in an eclipse for around 20 minuts: batteries are thus needed. Two litium-ions battery of 55 $Ah$ ech are present.The nominal polar orbit around Jupiter allows Sun pointing during the whole science operation Phase (NOME DELLA FASE A CUI CI RIFERIAMO).              INSERIRE RIFERIMENTI: VIDEO, SLIDES LAVAGNA, JUNO Mission
    \item \textbf{Providing the correct temperature range across the instruments during operation and the jurney}
    \newline The delicate suites of instrument present onboard JUNO require a very narrow range of temperature to operate, as 
    low as $\pm 1 K$. The MWR instruments infact needs is to measure with hight accuracy the microwawe emission from Jupiter. During the cruise phase JUNO will pass as close as 0.88 AU from the Sun, so the S/C will need to protect the P/L from the incoming heat. Passive solutions were preferred to mantain semplicity and reliability during the whole mission period. Juno's thermal control subsistem uses a design with heaters and louvers. It consists of an insulated louvered electronics vault atop a heated propulsion module. 
    LINK NASA JPL 
    \item \textbf{Shielding the instruments from the ursh envirorment of Jupiter}
    \newline To accomplish its goals, JUNO will need to cross the Jupiter radiation belts: a heavy shieldind structure is needed. Its task is to reduce the exposition level of the electronics by a factor of 800: S/C in LEO are needed to shield up to 0.65 $\mi T$, while the on Jupiter the expected intesity of the magnetic field is 518 $\mi T$. The measured value measured by JUNO at its perijove is 776 $\mi T$, 50 \% higher than expected. The voult in which all the electronics is preserved is cubed shaped and it is made of 1 centimeter thick titanium, 144 $Kg$ in total. Moreover, startrackers are 
\end{enumerate}  