\section{High level goals}
\label{sec:goals}

Through an analysis of the mission and payload, the main goals of the mission can be highlighted. 

\begin{enumerate}
    \item \textbf{How did Jupiter form and influence the solar system?} \cite{2006_overview} \cite{video_1h}
    
    Since Jupiter is the biggest planet of the solar system, it has influenced the formation of all other planets. Its composition has remained unchanged ever since, making it like a time capsule: understanding how and where it formed could give knowledge on Earth and the whole solar system’s origin, evolution and characteristics. 
    \item \textbf{What's Jupiter's deep structure?} \cite{video_1h} \cite{Overview_Juno}
    
    One important aspect of the mission is the analysis of Jupiter's deep structure through the measurement of radiations, magnetic and gravitational fields. This allows to comprehend whether or not the planet has a solid nucleus, if so how large it is, and to analyze the supposed layer of metallic hydrogen, compressed so much that it loses its electrons creating a conducting layer.  Moreover, Juno will possibly reveal if Jupiter is rotating as a solid body or if the rotating interior is made up of concentric cylinders. %the juno mission 
    \item \textbf{What's the structure of Jupiter's atmosphere?} \cite{video_1h} \cite{Juno_mission}

    One of the mission's goals is to study the composition and dynamics of Jupiter's atmosphere, composed by stripes and dots made of different gasses and vapors, including water, whose percentage has to be defined. A significant aspect of the analysis is the great red spot, a swirling mass of gas bigger than Earth, which resembles a hurricane but is very different in the way it works. The movement of stripes and dots is dictated by the weather, characterized by lighting an thunderstorms, which are observed by Juno. 
    \item \textbf{What do auroras look like and what are the physical processes generating them?} \cite{video_1h} \cite{Juno_mission}
    
    Juno's orbit is designed to be polar, to allow the observation of Jupiter's poles and the analysis of its auroras, representative of the interaction between charged particles and the atmosphere. Studying this phenomenon allows a better understanding of the atmospheric composition and the magnetic field's structure and extension.
    \item \textbf{What do the poles look like?} \cite{video_1h}
    \label{goal 5}
    
    One of Juno's side goals is the observation of Jupiter's poles, which had never been possible before because of the absence of a polar orbiting spacecraft. This also increments the public's involvement in the mission.

\end{enumerate}