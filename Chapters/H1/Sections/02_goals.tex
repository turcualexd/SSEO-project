\section{High level goals}
\label{sec:goals}

Juno is a NASA spacecraft orbiting Jupiter. Built by Lockheed Martin and operated by NASA, it was launched by Atlas V551 on $5^{th}$ August 2011. After 5 years, during which many manoeuvres occurred, including an Earth flyby, Juno entered a polar orbit around Jupiter and started its observation, which lasts to this day. Its aim is to study the planet to understand its composition and evolution, analyzing its gravitational and magnetic fields and its atmosphere dynamics. The mission should have ended in 2017, but is still ongoing and it will end in a de-orbit destroying the spacecraft into the planet's atmosphere to avoid contaminating the environment. 


Through an analysis of the mission and payload, the main goals of the mission can be highlighted.


LA TABELLA CON I GOALS O I SCIENTIFIC OBJECTIVES? CHE PERO SONO PRATICAMENTE UGUALI PER TUTTO

\begin{enumerate}
    \item How did Jupiter form and influence the solar system?

    Since Jupiter is the biggest planet of the solar system, it has influenced the formation of all other planets. Its composition has remained unchanged ever since, making it like a time capsule: understanding how and where it has formed can give knowledge on Earth and the whole solar system’s origin, evolution and characteristics.

 
    \item What's Jupiter's deep structure?
    
    One important aspect of the mission is the analysis of Jupiter's deep structure through the measurement of radiations, magnetic and gravitational fields. This allows to comprehend wether or not the planet has a solid nucleus, if so how large it is, and to analyze the supposed layer of metallic hydrogen, compressed so much that it loses its electrons creating a conducting layer.  Moreover, Juno will possibly reveal if Jupiter is rotating as a solid body or if the rotating interior is made up of concentric cylinders.

    \item What's the structure of Jupiter's atmosphere?
    
    One of Juno's goals is to study the composition and dynamics of Jupiter's atmosphere, composed by stripes and dots made of different gasses and vapors, including water, whose percentage has to be defined. A significant aspect of the analysis is the great red spot, a swirling mass of gas bigger than Earth, which resembles a hurricane but is very different in the way it works. The movement of stripes and dots is dictated by the weather, characterized by lighting an thunderstorms, which are observed by Juno.

    \item What do the poles look like and what are the physical processes generating auroras?
    
    The observation of Jupiter's poles had never been possible before because of the absence of a polar orbiting spacecraft. Thanks to Juno we have been able to study Jupiter's auroras, representative of the interaction between charged particles and the atmosphere. This can allow a better understanding of the atmospheric composition and the magnetic field's structure and extension.

\end{enumerate}

\begin{table}[H]
    \renewcommand{\arraystretch}{1.5}
    \centering
    \begin{tabularx}{\linewidth}{|X|X|}
        \hline
        \textbf{Guiding questions} & \textbf{Science objectives} \\
        \hline
        How did Jupiter form and influence the solar system? & Map Jupiter's composition and image the whole planet \\ 
        \hline
        What's Jupiter's deep structure? & Analyze gravitational field, magnetic field and radiations \\
        \hline
        What's the structure of Jupiter's atmosphere? & Screen the planet, analyze atmospheric processes and emissions \\
        \hline
        What do the poles look like and what are the physical processes generating auroras? & Image Jupiter, study interactions between atmosphere and magnetic field \\
        \hline
    \end{tabularx}
    \caption{Guiding questions for high level goals}
    \label{table:goal_questions}
\end{table}