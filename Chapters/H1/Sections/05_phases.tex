\section{Main mission phases}
\label{sec:phases}

The JUNO mission was divided into five phases: Launch, Cruise, Insertion, Science Operations and De-orbit.
\begin{enumerate}
    \item \textbf{Launch} 
    

    The launch took place on August 5 2011, 16:25 UTC from Cape Carnival, in Florida. The selected launcher was Atlas V551, which injected the spacecraft in the desired trajectory.
    This phase began with the ignition of the booster engine system, and ended with spacecraft separation from the upper stage, about 54 minutes after. The solar panels deployment was performed about five minutes after the spacecraft separation, and it took approximately five minutes.

    \item \textbf{Cruise}
    
    At the beginning of this phase, the spacecraft was injected in an interplanetary trajectory with the aim of reaching Jupiter. The cruise had a duration of about five years, during which two deep space manoeuvres and an Earth fly-by were performed.
    All manoeuvres will be better described in \autoref{sec:mission_analysis}. This phase also included instruments testing and verification, to ensure they were functioning properly and ready for the usage during the mission. This phase was also characterized by initial science observations of Jupiter.

    \item \textbf{Insertion}
    
    This phase began four days before the start of orbit insertion manoeuvre and ended one hour after the start of the orbit insertion manoeuvre. The latter occurred at closest approach to Jupiter and slowed the spacecraft enough to let it be captured by Jupiter in a 53 days period orbit.
    The Jupiter orbit insertion burn was performed by the main engine, and it lasted 30 minutes. After the burn, the spacecraft was in a polar orbit around Jupiter.
    The 53-days orbit provided substantial propellant savings with respect to the direct insertion in the operational orbit.

    \item \textbf{Insertion}
    
    Before starting with the science operations, it was planned to reduce the period of the orbit from 53 to 14 days.
    However, due to problems with helium valves in the propulsion system, it was decided to remain in the 53-day orbit. The Juno polar, highly eccentric orbit, was designed to facilitate the close-in measurements and to minimize the time spent in the Jupiter radiation belts.

    \item \textbf{De-orbit}
    
    The de-orbit phase occurs during the final orbit of the mission. The latter was designed to satisfy NASA's planetary protection requirements and ensure that JUNO doesn't impact any of Jupiter's moons. A de-orbit burn will be performed, placing the spacecraft on a trajectory towards Jupiter surface.
    JUNO was not designed to operate in the atmosphere and will burn up.
    
     
\end{enumerate}