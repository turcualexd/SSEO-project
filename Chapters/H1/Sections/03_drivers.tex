\section{Mission drivers}
\label{sec:drivers}

Being Juno an interplanetary mission starting from a distance of around 1 AU, with a final nominal distance from the Sun of 5.2 AU, and operating in an highly radiation intense environment, the following drivers have been identified: 

\begin{enumerate}
    \item \textbf{Using proven technologies} THE Juno Mission

    The total program is financed with 1.1 Billion \$ for 74 months from the launch date and includes development of the S/C, science instruments, launch services, mission operations, science processing and relay support. The simplicity and the need of proven technologies was thus fundamental. The S/C is mainly maintained stable during the manuevers thanks to its spin, raised to 5 $RPM$ from 2, nominal condition during science operations, reducing the need of active stabilization methods.

    \item \textbf{Providing enough electricity during the duration of the mission} 
    \label{sec: solar drivers}
    
    The journey of Juno is long and passes through different regions of the solar system. Solar panels were chosen to provide electric energy across the mission over a nuclear source, since it has been decided that it was better to advance technology of solar cells rather than developing a new reactor. It is the first S/C to operate with solar panels at such distance from the Sun. The system needed is thus oversized at 1 AU: the solar radiation on Jupiter is in fact up to 96\% lower than on Earth. Furthermore the operations are scheduled to begin around 5 years into the mission, so degradation of the solar cells must be taken into account. The final design consists in 3 solar arrays, 2 are 9 by 2.65 $m$ each, meanwhile the third one is  shorter at about and only 2 $m$ wide, resulting in about 60 $m^2$ and granting a maximum power of 14 $KW$ around Earth and up to 500 $W$ around Jupiter. This configurations is needed to mount the MAG faraway from the electronics and store everything correctly inside the fairing. Its solar panels are mounted on the side of hexagonal body of the S/C at 120\textdegree \; one by the other and are deployed after second stage separation. The spacecraft is spinning at around 1.4 $RPM$ during this phase and the deployment of the solar arrays slows it down. A crucial element must be considered during the cruise phases: nominally the high gain antenna, that is mounted with the same positive direction of the solar arrays, is doing Earth pointing to communicate data but, at some point during the cruise phase, due to the proximity of the Sun, thermal requirements dictate a change in attitude and thus the spacecraft won't be pointing directly the Sun. Moreover, since the fly-by around Earth is done to gain $\Delta V$, the S/C will be in an eclipse for around 10 minutes: attention must be paid in battery sizing. Two lithium-ions battery of 55 $Ah$ each are present to make sure power is always provided. The nominal polar orbit around Jupiter allows Sun pointing during the majority of  nominal Science Operation phase.              INSERIRE RIFERIMENTI: VIDEO, SLIDES LAVAGNA, JUNO Mission, batteries
    \item \textbf{Shielding the instruments from the harsh environment of Jupiter}
    
    To accomplish its goals, Juno will need to cross the Jupiter radiation belts: a heavy shielding structure is needed. The magnetosphere represents a great challenge for Juno: the value of the magnetic field measured at its perijove is 776 $\mu T$, 50 \% higher than expected. The main issue with Juno orbit is represented by the ionizing particles present in the belt around Jupiter: with measured value up to of tens and hundreds of MeV ions located between 2 and 4 $R_{J}$, order of GeV were expected under 2 $R_{J}$, where Juno should pass through to reach a lower altitude, thanks to its highly elliptical orbit, where radiations are lower, to perform science. The vault in which all the electronics is preserved is cubed shaped and it is made of 1 $cm$ thick titanium alloy, 144 $Kg$ in total. The top deck of Juno is planned to receive a radiation dose of 22 $Mrad$. Moreover, star trackers are also heavily shielded. % All these attentions are needed to allow science operation and to preserve the integrity of the S/C during its lifetime. 
    \item \textbf{Maintaining communication during the journey and the science operations}
    
    %Given the amount of data collected by the instruments, a continuos link with ground operations is needed. 
    The attitude of the S/C is defined in a way to point Earth during most of the cruise and science operations. This configurations, given the distance from the Sun and the Earth, grants also a sufficient inclinations of the solar panels with respect to the Sun to provide enough electric power. The ground equipment used by Juno is NASA's DSN.  
\end{enumerate}