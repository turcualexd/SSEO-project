\section{Payload analysis}
\label{sec:payload}

\subsection{Instruments overview}
As previuosly described in section \ref{sec:goals} the mission scientific goals are 
quite numerous and diverse. Thus, to achieve all of them the payload consists of 
several instruments, 9 to be precise, covering a wide spectrum of experimantations.

% Forse inserire massa, potenza e dati prodotti totali

Here we have a brief overview of all the singular instruments, unless otherwise 
specified only the sensors are mounted on the exterior of the spacecraft while all 
the relevant electronics are located inside the radiation vault. 

\begin{itemize}
    \item \textbf{Magnetometer (MAG):} As the name implies its objective is to 
    accurately measure Jupiter's magnetic field, achieved by employing a fluxgate 
    magnetometer, a scalar helium magnetometer and star cameras. All the sensors 
    are mounted on the magnetometer boom located at the end of one of the solar 
    array wings to reduce the interference from the spacecraft itself. Even then the 
    presence of two magnetometers allows to subtract this contribution from the 
    measurment.
    
    \item \textbf{Microwave Radiometer (MWR):} It consists of six antennas which 
    measure six different frequencies (600 MHz, 1.2 GHz, 2.4 GHz, 4.8 GHz, 9.6 GHz 
    and 22 GHz) in order to investigate the Jovian atmosphere below the visible 
    external layer. A key objective of this analysis is also the determination of 
    the abundace of water inside the planet. The antennas are mounted on two sides 
    of the exagonal prism that constitutes the main body of the spacecraft.
    
    \item \textbf{Gravity science:} It's quite a unique instrument as it's composed
    both by a space and a ground elements. The space segment is tasked with 
    amplifying and sending back radio signals which are instead generated and 
    received by the ground station. By measuring the doppler shift in the returning
    signal from Juno is possible to characterize Jupiter's gravitational field. Thus
    the instrument hardware mainly consists in the high gain antenna mounted on top 
    of the radiation vault on the main deck pointing in the same direction as the 
    solar arrays.
    
    \item \textbf{Jupiter Energetic-particle Detector Instrument (JEDI):} It detects
    high energy electrons and ions present in the Jovian magnetosphere which are
    discriminated by composition. Each sensor is characterized by six electron and 
    six ion viewing directions that together cover a \(12^{\circ} \times 
    160^{\circ}\) field of view. In total three sensors are present on Juno, two
    arranged to obtain an almost compete \(360^{\circ}\) view perpendicular to 
    the spacecraft spin axis whle the third one is instead aligned with it to 
    achieve a full scan of the sky over one spin period. As the JEDI sensors are
    self-contained no electronic hardware is present in the radiation vault.
       
\end{itemize}