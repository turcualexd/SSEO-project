\section{High level goals}
\label{sec:goals}
Juno is a NASA spacecraft orbiting Jupiter. Built by Lockheed Martin and operated by NASA, it was launched by Atlas V551 on 5 August 2011. After 5 years, during which many manoeuvres occurred, including an Earth flyby, Juno entered a polar orbit around Jupiter and started its observation, which lasts to this day. Its aim is to study the planet to understand its composition and evolution, analyzing its gravitational and magnetic fields and its atmosphere dynamics. The mission should have ended in 2017, but is still ongoing and it will end in a deorbit destroying the spacecraft into the planet's atmosphere to avoid containating the environment. 
\\

Through an analyisis of the mission and payload, the main goals of the mission can be highlighted. \\


SE VOGLIAMO METTERE IN UNA TABELLA NON SO COME SI FA \\

1. Jupiter's origin and influence. \\
Jupiter is the biggest planet in the solar system, so it's highly possible that it has influenced the formation of other planets with its gravity. Its composition has remained unchanged since the beginning of solar system, making it like a time capsule. Understanding how and where Jupiter has formed can give knowledge on Earth and the whole solar system's origin and characteristics.\\

2. Insights on deep structure and atmospheric processes\\
One important aspect of the mission is the poles observation, to study Jupiter's auroras, representative of the interaction between charged particles and the atmosphere. This can allow a better understanding of the atmospheric composition and the magnetic field's structure, verifying the existence of a layer of 'metallic hydrogen' (compressed so much that it loses its electrons creating a conducting layer). Completing the analysis is the study on Jupiter's deep structure using the measurement of radiations, to comprehend wether or not it has a solid nucleus and the percentage of water in the planet.\\

3. Study of Jupiter's stripes and spots\\
Jupiter is well known for its stripes and dots of gasses. The most evident one is the great red spot, a swirling mass of gas bigger than Earth, which resembles a hurricane but is very different in the way it works. The movement of stripes and dots is dictated by the weather, characterized by lighting an thunderstorms. Jupiter gets less sunlight than Earth, so the dynamics of these phenomena is different and to be studied. 






