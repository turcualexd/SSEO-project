\section{Payload analysis}
\label{sec:payload}

\subsection{Instruments overview}
As previously described in \autoref{sec:goals} the mission scientific goals are quite numerous and diverse. Thus, to achieve all of them the payload consists of several instruments, 9 to be precise, covering a wide spectrum of experimentation. In its entirety the payload has a mass of around $174$ kg and consumes approximately $125$ W of power excluding the Gravity science experiment \cite{Key_requirements}.
Here we have a brief overview of all the singular instruments where, unless otherwise specified, only the sensors are mounted on the exterior of the spacecraft while all the relevant electronics are located inside the radiation vault. 

\begin{itemize}
    \item \textbf{Magnetometer (MAG):} As the name implies its objective is to accurately measure Jupiter's magnetic field, achieved by employing two flux-gate magnetometers, a scalar helium magnetometer and two star cameras. All the sensors are mounted on the magnetometer boom located at the end of one of the solar array wings to reduce the interference from the spacecraft itself. Even then the presence of two magnetometers allows to subtract this contribution from the measurement.
    
    \item \textbf{Microwave Radiometer (MWR):} It consists of six antennas which measure six different frequencies ($600$ MHz, $1.2$ GHz, $2.4$ GHz, $4.8$ GHz, $9.6$ GHz and $22$ GHz) in order to investigate the Jovian atmosphere below the visible external layer. A key objective of this analysis is also the determination of the abundance of water inside the planet. The antennas are mounted on two sides of the hexagonal prism that constitutes the main body of the spacecraft relying on its spin to survey Jupiter.
    
    \item \textbf{Gravity science:} It's quite a unique instrument as it's composed both by a space and a ground elements, which mainly consists with the telecommunication systems of both the spacecraft and ground stations. This is because this experiment is based on measuring the doppler shift in the returning signal from Juno which allows to characterize Jupiter's gravitational field. Thus the instrument can't really be separated from the telecommunication hardware which is the reason its weight and power requirement were omitted in the previously shown totals.
    
    \item \textbf{Jupiter Energetic-particle Detector Instrument (JEDI):} It detects high energy electrons and ions present in the Jovian magnetosphere which are discriminated by composition. Each sensor is characterized by six electron and six ion viewing directions that together cover a $160^{\circ} \times 12^{\circ}$ field of view. In total three sensors are present on Juno, two arranged to obtain an almost compete $360^{\circ}$ view perpendicular to the spacecraft spin axis while the third one is instead aligned with it to achieve a full scan of the sky over one spin period. As the JEDI sensors are self-contained units no electronic hardware is present within the radiation vault.

    \item \textbf{Jovian Auroral Distribution Experiment (JADE):} It detects low energy electrons and ions with the same goal of characterizing the magnetosphere as JEDI. The instrument comprises of three identical electron energy per charge analyzers (JADE-E) and a single ion mass spectrometer (JADE-I). The electron sensors are located on the three sides of the  spacecraft that do not house the solar arrays pointing outwards, to again obtain a complete view normal to the spin axis. The spectrometer field of view, instead, contains the spin axis and like the third JEDI sensor it scans all the sky over a full rotation.
    
    \item \textbf{Ultraviolet Spectrograph (UVS):} This instrument images and measures the spectrum of the Jovian aurora in order to understand its morphology and source. The chosen ultraviolet range of $68 \div 210$ nm covers all of the most important UV emissions form the aurora, mainly the H Lyman series and longer wavelengths from hydrocarbons. The sensor is mounted on the side of Juno, relying once more on the spinning of the spacecraft to achieve a full sweep of the planet. 
    
    \item \textbf{Radio and Plasma Waves (Waves):} Its objective is to study both components of the electromagnetic field generated by plasma and radio waves inside the polar regions of Jupiter's magnetosphere to understand its interaction with the atmosphere and magnetic field. To detect the electric component a V-shaped dipole antenna is used, while for the magnetic component a much smaller magnetic search coil is employed. Both sensors cover a vast range of frequencies, namely from $50$ Hz up to $40$ MHz.

    \item \textbf{Visible-spectrum Camera (JunoCam):} It's designed to provide highly detailed color images of Jupiter to help and support public engagement of the mission without any real scientific purpose. The instrument is thus only comprised of the camera itself, mounted on the side of the spacecraft, and all the necessary electronics which, given the less critical objective and relaxed radiation tolerance requirements, aren't housed in the radiation vault.

    \item \textbf{Juno Infra-Red Auroral Mapper (JIRAM):} It's an infra-red imager and spectrometer that studies the Jovian atmosphere in the $2 \div 5 \; \mu$m range complementing both the atmospheric and magnetospheric experiments. This instrument is also completely housed outside of the radiation vault since it is a late addition after mission selection, reason for both the relaxed radiation requirements and less than ideal positioning of the sensor on the aft deck of the spacecraft.
       
\end{itemize}

\pagebreak
All the instruments and their positions can be seen in \autoref{fig:Strumenti Juno}.
\cfig{Strumenti Juno}{0.8}{Positioning of the instruments on the spacecraft}

\subsection{Payload and Goals correlation}
There is a notable overlap in the main objectives of the payload instruments, both in the sense that multiple ones collaborate towards a single scientific goal, but also in the sense that a single instrument can address multiple goals. All of these relations are exemplified in \autoref{table:Payload-goals correlation}.

\begin{table}[H]
    \renewcommand{\arraystretch}{1.5}
    \centering
    \begin{tabularx}{\linewidth}{|X|X|X|}
        \hline
        \textbf{Guiding questions} & \textbf{Science objectives} & \textbf{Measurements objectives} \\
        \hline
        \hline
        How did Jupiter form and influence the solar system? & Determine Jupiter's inner composition & Composition analysis: MWR \\ 
        \hline
        What's Jupiter's deep structure? & Analyze gravitational and magnetic field, measure water abundance in the planet & Gravitational field analysis: Gravity science \newline Magnetic field analysis: MAG \newline Water abundance measurements: MWR \\
        \hline
        What's the structure of Jupiter's atmosphere? & Analyze atmospheric composition and dynamics  & Atmospheric composition determination: MWR \newline Atmospheric dynamics study: JIRAM \\
        \hline
        What do the poles look like and what are the physical processes generating auroras? & Image auroras, study interactions between atmosphere and magnetic field, characterize the magnetosphere in the polar regions & Imaging auroras: UVS, JIRAM \newline Atmosphere-magnetic field interaction: JIRAM \newline Characterize the magnetosphere: Waves, JADE, JEDI \\
        \hline
    \end{tabularx}
    \caption{Mission goals and instrument objectives correlation}
    \label{table:Payload-goals correlation}
\end{table}

It can be noted that the JunoCam instrument doesn't appear in the table since it's not part of the scientific goals of the mission, as previously mentioned in its description.

\subsection{Payload and Phases/ConOps correlation}
Another high-level correlation can be highlighted between the mission phases/ConOps and the activities of the payload as shown in \autoref{table:Payload-phases correlation} \cite{Juno_launch}.

\begin{table}[H]
    \renewcommand{\arraystretch}{1.5}
    \centering
    \begin{tabularx}{\linewidth}{|X|X|}
        \hline
        \textbf{Mission phases} & \textbf{Payload activities} \\
        \hline
        \hline
        LEOP & Mag boom is deployed together with solar arrays \\
        \hline
        Cruise & Instruments checks are performed regularly and the high gain antenna (used for Gravity science) is calibrated and aligned \\
        \hline
        Jupiter approach and insertion & Final instruments checks are carried out together with some initial scientific observations of Jupiter \\
        \hline
        Science operations & Complete nominal operation of the payload with observations divided between Gravity science passes (Earth pointing) and MWR passes (Nadir pointing) \\
        \hline
        De-orbit & No planned payload operations \\
        \hline
    \end{tabularx}
    \caption{Mission phases/ConOps and Payload activities correlation}
    \label{table:Payload-phases correlation}
\end{table}