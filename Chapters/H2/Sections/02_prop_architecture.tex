\section{Propulsion system architecture}
\label{sec:prop_architecture}

% descrizione del sdr 

The spacecraft axes are defined as shown in \autoref{fig:axis}: the +Z-axis is aligned with the HGA, the +X-axis is aligned with the MAG boom while the +Y-axis is in the direction of cross product between +Z-axis and +X-axis.

\cfig{axis}{0.85}{Axis description}

\subsection{Main Engine and RCS}
\label{sec:me and rcs}
% diciamo cosa abbiamo a bordo, tank e nozzles
Juno is equipped with a dual mode propulsion subsystem: the bi-propellant ME uses the hypergolic couple hydrazine and nitrogen tetroxide($N_2H_4 - N_2O_4$).
The ME is a Leros-1b, built by Nammo \cite{Leros}, and produces about 662 N of thrust with a $I_s$ of 318.6 $s$. This engine is utilized during the DSMs, JOI and PRM. It is mounted inside the body of the spacecraft along the -Z-axis, centred between the propellant tanks and under the electronic vault. This solution has probably been adopted due to space requirements inside the Atlas V fairing and safety precautions during the cruise phase. The ME is also shielded by a hatch that opens when a manuever is needed.
The RCS uses only hydrazine as mono-propellant and it is used for TCMs, attitude control and SK. The whole RCS is composed by four REMs, each of them mounted along the $\pm$Y-axis, as shown in \autoref{fig:spin_direction}. Each pylon houses three thrusters providing a thrust of 4.5 N with $I_s$ of 220$s$. 
% The REMs are installed on pylons, raised respectively by 74 cm on the top deck and about 26 cm on the lower deck as shown in \autoref{fig:foto_thruster}. Moreover, always with the need of limiting the interaction of the exhaust gasses with the on board instruments, mainly mounted on the top deck, and solar panels, the axial thrusters are canted 10 \textdegree  away from the Z-axis while the lateral thrusters are canted 5\textdegree away from the X-axis and 12.5\textdegree toward the Z-axis.

\cfig{spin_direction}{0.65}{Forward and Aft deck view}

The REMs are installed on pylons, raised respectively by 74 cm on the top deck and about 26 cm on the lower deck as shown in \autoref{fig:foto_thruster}. Moreover, with the need of limiting the interaction of the exhaust gasses with the on board instruments, mainly mounted on the top deck, and solar panels, the axial thrusters are canted 10\textdegree away from the Z-axis while the lateral thrusters are canted 5\textdegree away from the X-axis and 12.5\textdegree toward the Z-axis \cite{junno_inner}.
\subsection{Maneuver Implementation Modes}
Juno's manuever can be performed in two different modes: $vector\;mode$ and $turn-burn-turn$.

The $vector\;mode$ consists of separated and coordinated axial and lateral burns from RCS thrusters, with the lateral thrusters denominated by letter "L" and axial thrusters denominated by letter "A" in both \autoref{fig:spin_direction} and \autoref{fig:foto_thruster}. As seen in \autoref{sec:me and rcs} the thrusters are not exactly perpendicular one to the other, so during a $vector\;mode$ manuever an induced axial $\Delta V$ is generated and must be compensated.

\cfig{foto_thruster}{0.7}{RCS fordward mount}

The $turn-burn-turn$ mode consists in a sequence of RCS and ME burns: first the spacecraft slews to the design spinning rate of 5 RPM, than the ME is ignited and the manuever is performed. At last, the spacecraft slews back to its nominal spinning rate. This mode is used during all the ME burns. In this kind of manuever the RCS uses the "L" thrusters on the REMs. 

\subsection{Tanks}

Juno's tanks are equally distributed throughout its hexagonal shaped body. 4 tanks are needed to store hydrazine and 2 tanks are needed to store the oxidizer. As can be seen from \autoref{fig:foto_thruster}, the oxidizer ones (green tanks) are located along the X-axis, while fuel ones (blue ones) are placed in the remaining bays. The 2 tanks containing the supercritic helium needed to pressurize the propellant system are placed below the oxidizer tanks. A system of valves regulates the pressure inside the tanks to allow nominal operations of the ME and RCS. MEFs are planned to flush the main propellant line and to increase the precision of the different manuevers. 
In order to accomplish its mission, Juno holds about 2.000 Kg of propellant: about 1280 kg of fuel and 720 kg of oxidizer \cite{junno_inner}. A more detailed analysis of tanks will be conducted in QUI METTIAMO IL RIFERIMENTO AL SIZING
 
% riferimento alla section con il dimensionamento 











