\section{Mission analysis and \texorpdfstring{$\Delta V$}{Delta-V} budget}
\label{sec:ma_and_dv}
The mission analysis previously described can be split into two macro-categories:
\begin{itemize}
    \item the interplanetary transfer, including the Earth fly-by;
    \item the planetary phase around Jupiter.
\end{itemize}
The main objectives for the mission analysis were to keep the overall launch energy $C_3$ and determinstic $\Delta V$ as low as possible, compliant with the constraints imposed by the navigation and spacecraft requirements. Regarding the interplanetary transfer, two main options were avialable, both including and EGA. The first option, named as $2-\Delta V EGA$, had a launch window timeframe in October-November 2011. However, this option was discarded since the approach angle at Jupiter would have resulted in a latitude farther away from the equator. This would have brought to higher radiation levels, hence a reduced time available for the science operations. The second option, named as $2+\Delta V EGA$, had a launch window timeframe in August 2011 and it ended up being the chosen one. A viable back-up of this transfer would have happened in October 2012. 







$2+\Delta V EGA$, made up by a DSM (successively split in 2) and the EGA, was chosen.






\cite{fact_sheet}
