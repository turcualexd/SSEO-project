\section{Mission analysis and \texorpdfstring{$\Delta V$}{Delta-V} budget}
\label{sec:ma_and_dv}
The mission analysis previously described could be split into two macro-categories:
\begin{itemize}
    \item from launch to the interplanetary transfer, including DSMs and the Earth fly-by ;
    \item the planetary phase around Jupiter.
\end{itemize}
The main objectives for the mission analysis were to keep the overall launch energy $C_3$ and determinstic $\Delta V$ as low as possible, compliant with the constraints imposed by the navigation and spacecraft requirements. Regarding the interplanetary transfer, two main options were avialable, both including an EGA. The first option, named as $2-\Delta V EGA$, had a launch window timeframe in October-November 2011. However, the latter was discarded since the approach angle at Jupiter would have resulted in a latitude farther away from the equator. This would have brought to higher radiation levels, hence a reduced time available for the science operations. The second option, named as $2+\Delta V EGA$, had a launch window timeframe in August 2011 and it ended up being the chosen one. A viable back-up of this transfer would have happened in October 2012, since the basic features of $2+\Delta V EGA$ repeat every 13 months.
Regarding the interplanetary trajectory constraints, they could be divided in three categories:
\begin{itemize}
    \item \textbf{Launch energy $C_3$ and timing constraints.} Fixing an inital value for the launch energy provided by the launcher, different possibilites of launch date could be analyzed. Every launch date defined a trajectory that was characterized by a required $C_3$ and determinstic $\Delta V$. The maximum $C_3 = 31.1 km^2 / s^2$ was defined by the Atlas V551 launcher \cite{atlasV_juno}. Analyzing Juno's ephemerides for the actual launch date, the calculated value is $C_3 = 31.08 km^2 / s^2$. The trajectory reconstructed through the optimization problem and explained (subsec), revealed a value of $C_3 =  29.3413 km^2 / s^2$, with departure date on 18/08/2011. Restricting the launch window domain around 05/08/2011 (the actual one) the value is $C_3 = 30.4047 km^2 / s^2$. In the non-restricted window for departure, the trend of the decreasing $C_3$ over time can be evaluated in the work of Kowalkowski and Lam\cite{launch_period}.
    \item \textbf{Interplanetary events} The milestones of this phase were DSMs and EGA. The fly-by was constrained to happen at a fixed altitude of $800 km$ well above ISS, but it could be lowered up to $500km$.(cerca fonte FB reale). However, the most challenging task was the selection of the DSM dates, in fact this choice would have affected the required launch $C_3$ and overall mission $\Delta V$. Moreover, DSM manuever required to be split into two equally lasting burns, separated by 2 days. This was required by engine capabilities, described in (autoref). As precaution, due to anomalous pressure and temperature values of the oxidizer feeding line during DSM1, the two manuevers ended up being performed two weeks apart. An additional problem arose from the $2+\Delta V EGA$ interplanetary structure and the need to have real-time visibility. This constraint was set up by imposing a SEP angle greater than 10\textdegree \;for acquisition of data, and SEP angle greater than 3\textdegree \;for execution of the maneuver. This difference was due to the need of seven days to plan the maneuver on ground after the necessary two days of data collection. In the eventuality of a burn before solar conjunction, an adequate time to re-try failed attempts was considered. (ELA angle?)
    \item \textbf{Jupiter arrival timing and geometry}
\end{itemize}
 












\cite{fact_sheet}
